%%%%%%%%%%%%%%%%%%%%%%%%%%%%%%%%%%%%%%%%%
% Masters/Doctoral Thesis
% LaTeX Template
% Version 2.5 (27/8/17)
%
% This template was downloaded from:
% http://www.LaTeXTemplates.com
%
% Version 2.x major modifications by:
% Vel (vel@latextemplates.com)
%
% This template is based on a template by:
% Steve Gunn (http://users.ecs.soton.ac.uk/srg/softwaretools/document/templates/)
% Sunil Patel (http://www.sunilpatel.co.uk/thesis-template/)
%
% Template license:
% CC BY-NC-SA 3.0 (http://creativecommons.org/licenses/by-nc-sa/3.0/)
%
%%%%%%%%%%%%%%%%%%%%%%%%%%%%%%%%%%%%%%%%%

%%%%%%%%%%%%%%%%%%%%%%%%%%%%%%%%%%%%%%%%%
% José Miguel García Benayas
% Universidad Rey Juan Carlos
% Móstoles Madrid España
% Escuela Técnica Superior de Ingeniería Informática
% Grado en Ingeniería del Software
% 2019
%%%%%%%%%%%%%%%%%%%%%%%%%%%%%%%%%%%%%%%%%

%-------------------------------------------------------------------------------
%	PACKAGES AND OTHER DOCUMENT CONFIGURATIONS
%-------------------------------------------------------------------------------

\documentclass[
12pt, % The default document font size, options: 10pt, 11pt, 12pt
%oneside, % Two side (alternating margins) for binding by default, uncomment to switch to one side
spanish, % ngerman for German
onehalfspacing, % Single line spacing, alternatives: onehalfspacing or doublespacing
%draft, % Uncomment to enable draft mode (no pictures, no links, overfull hboxes indicated)
%nolistspacing, % If the document is onehalfspacing or doublespacing, uncomment this to set spacing in lists to single
%liststotoc, % Uncomment to add the list of figures/tables/etc to the table of contents
%toctotoc, % Uncomment to add the main table of contents to the table of contents
parskip, % Uncomment to add space between paragraphs
%nohyperref, % Uncomment to not load the hyperref package
headsepline, % Uncomment to get a line under the header
%chapterinoneline, % Uncomment to place the chapter title next to the number on one line
%consistentlayout, % Uncomment to change the layout of the declaration, abstract and acknowledgements pages to match the default layout
%openany % Uncomment to delete white page after introduction chapter
]{MastersDoctoralThesis} % The class file specifying the document structure

\usepackage[utf8]{inputenc} % Required for inputting international characters
\usepackage[T1]{fontenc} % Output font encoding for international characters

\usepackage{mathpazo} % Use the Palatino font by default
\usepackage{amsmath} % More math functions

\usepackage[backend=bibtex,natbib=true]{biblatex} % Use the bibtex backend with the authoryear citation style (which resembles APA)

\addbibresource{main.bib} % The filename of the bibliography

\usepackage[autostyle=true]{csquotes} % Required to generate language-dependent quotes in the bibliography

\usepackage{listings} % Package to type bash commands

\usepackage{imakeidx} % Package to make an index of words used in the papers

\makeindex[program=makeindex,columns=2,intoc=true,options={-s index_style.ist}]

\usepackage[hidelinks]{hyperref} % Used to hide the link frames

\usepackage[ruled, vlined, spanish, onelanguage, linesnumbered]{algorithm2e} %for psuedo code

\usepackage{rotating} 
\usepackage{schemata}
\newcommand\diagram[2]{\schema{\schemabox{#1}}{\schemabox{#2}}}

\usepackage{graphicx}
\usepackage{float}
\usepackage{subfigure}
\usepackage{amssymb, amsmath, amsbsy} % simbolos
\usepackage{upgreek} % para poner letras griegas sin cursiva

\usepackage[automake]{glossaries} %Load glossaries package
\makeglossaries

\usepackage[bottom]{footmisc}


% Start - Convert csv to table %
\usepackage{booktabs} % For \toprule, \midrule and \bottomrule
\usepackage{siunitx} % Formats the units and values
\usepackage{pgfplotstable} % Generates table from .csv
\usepackage{longtable}

\sisetup{
	round-mode          = places, % Rounds numbers
	round-precision     = 4, % to 2 places
}
\pgfkeysifdefined{/pgfplots/table/output empty row/.@cmd}{
	% upcoming releases offer this more convenient option:
	\pgfplotstableset{
		empty header/.style={
			every head row/.style={output empty row},
		}
	}
}{
	% versions up to and including 1.5.1 need this:
	\pgfplotstableset{
		empty header/.style={
			typeset cell/.append code={%
				\ifnum\pgfplotstablerow=-1 %
				\pgfkeyssetvalue{/pgfplots/table/@cell content}{}%
				\fi
			}
		}
	}
}
% End - Convert csv to table %



\IncMargin{1.5em}
\SetNlSty{texttt}{(}{)}


\setcounter{secnumdepth}{4} % Depth of subsections

%-------------------------------------------------------------------------------
%	MARGIN SETTINGS
%-------------------------------------------------------------------------------

\geometry{
	paper=a4paper, % Change to letterpaper for US letter
	inner=2.5cm, % Inner margin
	outer=2.5cm, % Outer margin
	top=2.5cm, % Top margin
	bottom=2.5cm, % Bottom margin
	%showframe, % Uncomment to show how the type block is set on the page
}

%-------------------------------------------------------------------------------
%	THESIS INFORMATION
%-------------------------------------------------------------------------------

\thesistitle{Búsqueda del clique de ratio máximo mediante el algoritmo GRASP} % Your thesis title, this is used in the title and abstract, print it elsewhere with \ttitle
\supervisor{Dr. Jesús Sánchez-Oro Calvo\\Dr. Alfonso Fernández Timón} % Your supervisor's name, this is used in the title page, print it elsewhere with \supname 
%\cosupervisor{Dr. Alfonso Fernández Timón} % Your supervisor's name, this is used in the title page, print it elsewhere with \supname 
\examiner{} % Your examiner's name, this is not currently used anywhere in the template, print it elsewhere with \examname
\degree{Grado en Ingeniería del Software} % Your degree name, this is used in the title page and abstract, print it elsewhere with \degreename
\author{José Miguel García Benayas} % Your name, this is used in the title page and abstract, print it elsewhere with \authorname
\addresses{} % Your address, this is not currently used anywhere in the template, print it elsewhere with \addressname

\subject{} % Your subject area, this is not currently used anywhere in the template, print it elsewhere with \subjectname
\keywords{} % Keywords for your thesis, this is not currently used anywhere in the template, print it elsewhere with \keywordnames
\university{\href{https://www.urjc.es}{Universidad Rey Juan Carlos}} % Your university's name and URL, this is used in the title page and abstract, print it elsewhere with \univname
\department{} % Your department's name and URL, this is used in the title page and abstract, print it elsewhere with \deptname
\group{} % Your research group's name and URL, this is used in the title page, print it elsewhere with \groupname
\faculty{Escuela Técnica Superior de Ingeniería Informática} % Your faculty's name and URL, this is used in the title page and abstract, print it elsewhere with \facname

\AtBeginDocument{
\hypersetup{pdftitle=\ttitle} % Set the PDF's title to your title
\hypersetup{pdfauthor=\authorname} % Set the PDF's author to your name
\hypersetup{pdfkeywords=\keywordnames} % Set the PDF's keywords to your keywords
}

\begin{document}

\frontmatter % Use roman page numbering style (i, ii, iii, iv...) for the pre-content pages

\pagestyle{plain} % Default to the plain heading style until the thesis style is called for the body content

%-------------------------------------------------------------------------------
%	TITLE PAGE
%-------------------------------------------------------------------------------

\begin{titlepage}
\begin{center}

\vspace*{.01\textheight}
\begin{figure}
\begin{center}
\includegraphics[scale=0.8]{Figures/URJC_logo.pdf}
\end{center}
\end{figure}
{\Large \MakeUppercase{\facname}}\\[0.5cm]
{\large \MakeUppercase{\degreename}}\\[0.5cm] % Research group name and department name
{\large Curso académico 2019/2020}\\[1.5cm] % Date
%{\scshape\LARGE \univname\par}\vspace{1.5cm} % University name
{\large \textbf{TRABAJO FIN DE GRADO}}\\[0.5cm] % Thesis type

{\Large \textbf{\MakeUppercase{\ttitle}}}\\[5cm] % Thesis title

{\normalsize Autor:\\{\authorname}}\\[0.5cm]  % Author name - remove the \href bracket to remove the link

{\normalsize Tutores:\\{\supname}}\\ % Supervisor name - remove the \href bracket to remove the link
%{\Large Cotutor: {\cosupervisor}}\\[1cm]

%\includegraphics{Logo} % University/department logo - uncomment to place it

\end{center}
\end{titlepage}

%-------------------------------------------------------------------------------
%	DEDICATION
%-------------------------------------------------------------------------------

\dedicatory{
	\begin{flushright}
		\normalsize
		<<Victory is always possible for the person who refuses to stop fighting>>,\\
		Napoleon Hill
	\end{flushright}
}


%-------------------------------------------------------------------------------
%	ABSTRACT PAGE
%-------------------------------------------------------------------------------

\begin{abstract}
\addchaptertocentry{\abstractname} % Add the abstract to the table of contents
En la presente memoria de este Trabajo Fin de Grado se trata el problema del clique de ratio máximo también llamado Maximal Ratio Clique Problem (MRCP), el cual consiste en encontrar un subgrafo completo o clique, de ratio máximo. \\
A través de esta memoria se intenta explicar, de una manera clara, como se ha abordado el problema desde el diseño hasta la implementación, pasando por el estudio y comprensión de la metaheurística empleada, para la búsqueda optima de una solución factible. La implementación se ha realizado mediante el algoritmo Greedy Randomized Adaptive Search Procedure (GRASP) o Procedimiento de Búsqueda Voráz Aleatoria y Adaptativa, el cual ofrece buenas soluciones en un tiempo de computo razonable.

Este problema es una derivación del clásico Problema del Clique Máximo o Maximal Clique Problem (MCP) que consiste en encontrar el clique máximo en un grafo.

El MRCP tiene aplicaciones tan actuales como el machine-learning o las redes sociales, ya que un clique es un grupo de personas con intereses afines y por ello se ha intentado implementar un algoritmo que trate los datos de la manera más eficiente posible, de forma que se agilice el tratamiento de estos posteriormente.
\end{abstract}

%-------------------------------------------------------------------------------
%	ACKNOWLEDGEMENTS
%-------------------------------------------------------------------------------

\begin{acknowledgements}
\addchaptertocentry{\acknowledgementname} % Add the acknowledgements to the table of contents
Todo este trabajo no habría sido posible sin mi familia, sin su apoyo y cariño, en especial a Rebeca, la cual ha sido mi compañera de viaje y la luz que me ilumina.

Sin olvidar a mis amigos en la universidad, con quienes he disfrutado cada día en clase, y a mis tutores, Jesús y Alfonso, por darme la oportunidad de realizar este trabajo y de los que he aprendido mucho.
\end{acknowledgements}

%-------------------------------------------------------------------------------
%	LIST OF CONTENTS/FIGURES/TABLES PAGES
%-------------------------------------------------------------------------------

\tableofcontents % Prints the main table of contents

\listoffigures % Prints the list of figures

\listoftables % Prints the list of tables

%-------------------------------------------------------------------------------
%	ABBREVIATIONS
%-------------------------------------------------------------------------------

% Acronym definitions
\newglossaryentry{API}{name={API},description={Application Programming Interface}}
\newglossaryentry{MRCP}{name={MRCP},description={Maximum Ratio Clique Problem}}
\newglossaryentry{MCP}{name={MCP},description={Maximum Clique Problem}}
\newglossaryentry{MWCP}{name={MWCP},description={Maximum Weight Clique Problem}}
\newglossaryentry{GRASP}{name={GRASP},description={Greedy Randomized Adaptive Search Procedure}}
\newglossaryentry{IDE}{name={IDE},description={Integrated Development Environment}}
\newglossaryentry{TS}{name={TS},description={Tabu Search}}
\newglossaryentry{GLS}{name={GLS},description={Guided Local Search}}
\newglossaryentry{NM}{name={NM},description={Noising Methods}}
\newglossaryentry{SA}{name={SA},description={Simulated Annealing}}
\newglossaryentry{TAM}{name={TAM},description={Threshold Accepting Methods}}
\newglossaryentry{VNS}{name={VNS},description={Variable Neighborhood Search}}
\newglossaryentry{POPMUSIC}{name={POPMUSIC},description={Partial OPtimization Metaheuristic Under Special Intensification Conditions}}
\newglossaryentry{ILS}{name={ILS},description={Iterated Local Search}}
\newglossaryentry{FANS}{name={FANS},description={Fuzzy Adaptive Neighborhood Search}}
\newglossaryentry{HC}{name={HC},description={Heuristic Concentration}}
\newglossaryentry{AMS}{name={AMS},description={Adaptive Multi-Start}}
\newglossaryentry{MSM}{name={MSM},description={Multi-Start Methods}}
\newglossaryentry{CA}{name={CA},description={Cultural Algorithms}}
\newglossaryentry{GA}{name={GA},description={Genetic Algorithms}}
\newglossaryentry{MA}{name={MA},description={Memetic Algorithms}}
\newglossaryentry{PR}{name={PR},description={Path Relinking}}
\newglossaryentry{SS}{name={SS},description={Scatter Search}}
\newglossaryentry{ACO}{name={ACO},description={Ant Colony Optimization}}
\newglossaryentry{AT}{name={AT},description={Asynchronous Teams}}
\newglossaryentry{EDA}{name={EDA},description={Estimation Distribution Algorithms)}}
\newglossaryentry{SI}{name={SI},description={Swarm Intelligence}}
\newglossaryentry{GIL}{name={GIL},description={Global Interpreter Lock}}

%Print the glossary
\printglossary[title={Listado de abreviaciones}] %Generate List of Abbreviations

%-------------------------------------------------------------------------------
%	THESIS CONTENT - CHAPTERS
%-------------------------------------------------------------------------------

\mainmatter % Begin numeric (1,2,3...) page numbering

\pagestyle{thesis} % Return the page headers back to the "thesis" style

% Include the chapters of the thesis as separate files from the Chapters folder
% Uncomment the lines as you write the chapters

% Chapter 1: Introduction

\chapter{Introducción} % Main chapter title

\label{Chapter1}

%-------------------------------------------------------------------------------

% Define some commands to keep the formatting separated from the content
\newcommand{\keyword}[1]{\textbf{#1}}
\newcommand{\tabhead}[1]{\textbf{#1}}
\newcommand{\code}[1]{\texttt{#1}}
\newcommand{\file}[1]{\texttt{\bfseries#1}}
\newcommand{\option}[1]{\texttt{\itshape#1}}
\newcommand{\Mod}[1]{\ (\mathrm{mod}\ #1)}

%-------------------------------------------------------------------------------
En este capítulo se introduce el tema a tratar partiendo de conceptos previos que ayudarán al lector a entender mejor el desarrollo, siguiendo con la definición del problema y la motivación del mismo, y por último, se muestra una revisión del estado del arte relacionado con este problema.

\section{Estructura de la memoria}
La memoria de este trabajo final de grado se estructura de la siguiente manera:
\begin{itemize}
	\item Capítulo 1, en este capítulo se introduce el tema a tratar, describiendo el problema y el estado del arte del mismo, así como conceptos previos a tener en cuenta.
	\item Capítulo 2, en este capítulo se muestran los objetivos que se esperan alcanzar con este trabajo final de grado.
	\item Capítulo 3, en este capítulo se describe de manera algorítmica la metaheurística empleada para abordar el problema tratado.
	\item Capítulo 4, en este capítulo se explica el desarrollo de las funciones para obtener una solución al problema.
	\item Capítulo 5, en este capítulo se exponen los resultados recopilados durante el procesamiento del problema, así como el análisis de los mismos.
	\item Capítulo 6, en este capítulo se muestran las conclusiones finales obtenidas durante todo este trabajo final de grado.
\end{itemize}

\section{Conceptos previos}
Para comprender mejor todas las explicaciones que se van a exponer a lo largo de toda esta memoria, se van a describir algunos conceptos a continuación.

\subsection{Optimización combinatoria}
aaa

\subsection{Heurística y Metaheurística}
Se define heurística, según el Diccionario de la Real Academia de la Lengua Española \cite{rae-heuristica}, como ``técnica de la indagación y del descubrimiento'' y ``en algunas ciencias, manera de buscar la solución de un problema mediante métodos no rigurosos, como por tanteo, reglas empíricas, etc.''.

Aplicándolo a temas científicos se define como el proceso de creación de medios, estrategias y principios para alcanzar un objetivo eficaz al problema dado \cite{conceptodef-heuristica}, y en relación a esto, el término, fue acuñado por George Polya en su libro ``How to Solve It'' \cite{gpolya-book-1}, más tarde traducido a ``Cómo plantear y resolver problemas'' \cite{gpolya-book-2}.

Añadiendo al término heurística el prefijo ``meta'' procedente del griego, que significa ``más allá'' o ``nivel superior'', se puede definir metaheurística como el conjunto de procedimientos heurísticos combinados para obtener una solución, aunque no exacta si eficiente, a un problema que no tiene un algoritmo heurístico específico o su aplicación es ineficiente \cite{wiki-metaheuristica}. Este término lo acuño Fred Glover en sus trabajos sobre búsqueda tabú en 1986 \cite{fred-glover}.

Existen un gran variedad de algoritmo metaheurísticos que se pueden clasificar de diferentes maneras según el enfoque, en la figura \ref{fig:clasif-metahs} se muestra una de ellas basada en el número de soluciones.\\
\begin{figure}[H]
	\centering
	\diagram{Trayectoriales}{
		- \diagram{Búsqueda\\local}{
			- Con memoria: Búsqueda Tabú (\gls{TS})\\ 
			- \diagram{Estocástica}{
				- Búsqueda local guiada (\gls{GLS}) \\ 
				- Métodos ruidosos (\gls{NM}) \\ 
				- Recocido simulado (\gls{SA})\\
				- Métodos de aceptación de\\ umbral (\gls{TAM})\\
			}\\
			- \diagram{Por \\entornos}{
				- Búsqueda de entorno variable (\gls{VNS})\\
				- Optimización parcial bajo condiciones\\ especiales de intensificación(\gls{POPMUSIC})\\
			}\\
		}\\
		- \diagram{Búsqueda\\iterativa}{
			- Búsqueda local iterativa (\gls{ILS})\\
			- Búsqueda por entorno adaptativo borroso (\gls{FANS})\\
		}\\
		- \diagram{Búsqueda\\multi-arranque}{
			- Concentración heurística (\gls{HC})\\
			- Métodos multi-arranque adaptativos (\gls{AMS})\\
			- Métodos multi-arranque (\gls{MSM})\\
			- Procedimientos de búsqueda voráz,\\ aleatorizada y adaptativa (\gls{GRASP})\\
		}\\
	}
	\diagram{Poblacionales}{
		- \diagram{Combinación\\de soluciones}{
			- \diagram{Inspiración\\evolutiva}{
				- Algoritmos culturales (\gls{CA})\\
				- Algoritmos genéticos (\gls{GA})\\
				- Algoritmos meméticos (\gls{MA})\\
			}\\
			- \diagram{Sin inspiración}{
				- Re-encadenamiento de\\ caminos (\gls{PR})\\
				- Búsqueda dispersa (\gls{SS})\\
			}\\
		}\\
		- \diagram{Movimientos}{
			- Optimización por colonias de hormigas (\gls{ACO})\\
			- Equipos asíncronos (\gls{AT})\\
			- Algoritmos de estimación de distribución (\gls{EDA})\\
			- Inteligencia de enjambre (\gls{SI})\\
		}\\
	}
	\caption{Clasificación de metaheurísticas}
	\label{fig:clasif-metahs}
\end{figure}

Dos conceptos a tener en cuenta en un algoritmo metaheurístico son la intensificación o explotación en una región, de modo que una alta intensificación hará que el algoritmo realice una búsqueda más exhaustiva en esa región, y la diversificación o exploración de nuevas regiones del espacio de soluciones \cite{libro-metaheuristicas}.

\subsection{Clique}
\label{sec-clique}
El término clique se puede asemejar a un grupo de personas, las cuales serían los nodos o vértices de un grafo, a las que les unen los mismos intereses por algún tema en concreto, estas uniones serían las aristas que unen los nodos del grafo \cite{LUCE:1949}.

En teoría de grafos, un clique, es el subgrafo, perteneciente a un grafo, en el cuál todos sus nodos o vértices son adyacentes entre sí, es decir, todo par de nodos o vértices está conectados mediante una arista, en términos matemáticos se describe como, dado un grafo $G = (V, E)$ donde $V$ indica el conjunto de vértices del grafo y $E$ indica el conjunto de aristas del grafo \cite{web-clique}, un clique se define como:
\[
C \subseteq V(G) ~ \wedge ~ u, v ~ \epsilon ~ C  \wedge  u  \neq v \Rightarrow u, v ~ \epsilon ~ E(G)
\]

\begin{figure}[H]
	\centering
	\includegraphics{Figures/graph.pdf}
	\caption{Diagrama de un grafo.}
	\label{fig:graph}
\end{figure}

En la figura \ref{fig:graph} se muestra un grafo compuesto por el conjunto de nodos \\$V=\{A, B, C, D, E\}$, en este grafo se encuentran como indican las figuras \ref{fig:graph-cliques-2} y \ref{fig:graph-cliques-3} los posibles cliques de este grafo, donde $k$ indica el número de nodos del clique. Cabe resaltar, que los nodos por si solos también forman clique, pero no se ha incluido para no hacer más compleja la figura.

\begin{figure}[H]
	\centering	
	\subfigure[Clique A-B]{\includegraphics[width=0.275\textwidth]{Figures/graph-clique-AB.pdf}}
	\subfigure[Clique A-C]{\includegraphics[width=0.275\textwidth]{Figures/graph-clique-AC.pdf}}
	\subfigure[Clique A-D]{\includegraphics[width=0.275\textwidth]{Figures/graph-clique-AD.pdf}}
	\subfigure[Clique B-C]{\includegraphics[width=0.275\textwidth]{Figures/graph-clique-BC.pdf}}
	\subfigure[Clique B-D]{\includegraphics[width=0.275\textwidth]{Figures/graph-clique-BD.pdf}}
	\subfigure[Clique B-E]{\includegraphics[width=0.275\textwidth]{Figures/graph-clique-BE.pdf}}
	\subfigure[Clique C-D]{\includegraphics[width=0.275\textwidth]{Figures/graph-clique-CD.pdf}}
	\subfigure[Clique D-E]{\includegraphics[width=0.275\textwidth]{Figures/graph-clique-DE.pdf}}
	\caption{Cliques $k$ = 2 del grafo.}
	\label{fig:graph-cliques-2}
\end{figure}

\begin{figure}[H]
	\centering	
	\subfigure[Clique A-B-C]{\includegraphics[width=0.275\textwidth]{Figures/graph-clique-ABC.pdf}}
	\subfigure[Clique A-B-D]{\includegraphics[width=0.275\textwidth]{Figures/graph-clique-ABD.pdf}}
	\subfigure[Clique A-C-D]{\includegraphics[width=0.275\textwidth]{Figures/graph-clique-ACD.pdf}}
	\subfigure[Clique B-C-D]{\includegraphics[width=0.275\textwidth]{Figures/graph-clique-BCD.pdf}}
	\subfigure[Clique B-D-E]{\includegraphics[width=0.275\textwidth]{Figures/graph-clique-BDE.pdf}}
	\caption{Cliques $k$ = 3 del grafo.}
	\label{fig:graph-cliques-3}
\end{figure}

Es importante resaltar que un clique perteneciente a un grafo, no implica que sea máximo, puesto que un clique máximo es el clique el cuál no es posible ampliar, es decir, no se pueden añadir a este más adyacentes que cumplan con las restricciones necesarias para formar un clique y a su vez es el de mayor tamaño del grafo, como se muestra en la figura \ref{fig:max-clique}, a diferencia de un clique que se podría denominar simple  \cite{web-maximalclique}\cite{web-maximumclique}. Y en este  caso, se obtiene que:
\[
\omega(G) = 4
\]
donde $\omega$ denota el número de vértices del clique, su cardinalidad.
\begin{figure}[H]
	\centering
	\includegraphics{Figures/graph-clique-max.pdf}
	\caption{Clique máximo del grafo.}
	\label{fig:max-clique}
\end{figure}

\section{Definición y motivación del problema}

\subsection{Problema del clique de ratio máximo}
\label{intro-problema}
Como se introdujo en la sección \ref{sec-clique} el concepto clique en cuanto a teoría de grafos es fundamental y muy estudiado, y más concretamente la búsqueda del clique máximo dentro de un grafo, a este problema se le conoce como ``El problema del clique máximo'' o \gls{MCP} por sus siglas en ingles ``Maximum clique problem'', y es catalogado como un problema NP-completo, como se aprecia en la figura \ref{fig:problemas-np}.

\begin{figure}[H]
	\centering
	\includegraphics{Figures/problemas-np.pdf}
	\caption{Diagrama de problemas NP.}
	\label{fig:problemas-np}
\end{figure}

Los problemas denominados como NP, acrónimo de non-deterministic polynomial time o tiempo polinominal no determinista, en teoría de complejidad computacional, se les conoce como el conjunto de problemas que se pueden resolver en un tiempo polinómico por una máquina de Turing no determinista. Esta clasificación, además, contiene todos los problemas de tipo P y de tipo NP-completos como es el caso del problema del clique máximo.

Una variante de este problema es el llamado ``Problema de clique de peso máximo'' o \gls{MWCP} por sus siglas en inglés a Maximum weight clique problem, en el que se asocia un peso no negativo a cada vértice y cuyo objetivo es encontrar un clique con el máximo valor en la suma de los pesos de sus vértices. Este problema esta estrechamente ligado al problema tratado en este trabajo final de grado, el cual busca el clique de ratio máximo ya que si se asocian dos pesos no negativos a cada vértice se obtiene el problema del clique de ratio máximo \gls{MRCP} o ``Maximum Ratio Clique Problem''. En este problema se busca el clique máximo en un grafo con la mayor proporción de ratio. Esta proporción se define como las sumas de los pesos de los vértices.

\begin{equation*}
\frac{\sum_{i=1}^{n}p_ix_i}{\sum_{i=1}^{n}q_ix_i}
\end{equation*}

donde $p$ y $q$ son pesos no negativos asociados a cada vértice $i$, y $x$ se determina como:
\[
\diagram{$x_i=$}{
	1 : si el vértice $i$ forma parte del clique solución. \\
	0 : en otro caso
}
\]

Por lo tanto, el objetivo del problema es maximizar esta proporción como se expone en el modelo de ecuación \ref{eq:mrcp-max}, partiendo de un grafo simple no dirigido $G=(V, E)$, donde $V$ se asocia al conjunto de vértices pertenecientes al grafo, $\{v_1,\dots,v_n\}$, y $E$ es el conjunto de aristas que conectan los vértices del grafo, $\{v_i,~v_j\}$ tal que $i \neq j$ y $v_i,~v_j~\in V$, y suponiendo que los pesos asociados a cada vértice son positivos, se obtiene un clique máximo $\widehat{S}$, siempre y cuando se cumplan las restricciones \ref{eq:mrcp-rest1} - \ref{eq:mrcp-rest3}.

\begin{eqnarray}
\label{eq:mrcp-max} 
maximizar && f = \frac{\sum_{i=1}^{n}p_ix_i}{\sum_{i=1}^{n}q_ix_i} \\
\nonumber sujeto ~ a: \\
\label{eq:mrcp-rest1}
&& x_i + x_j \leqslant 1 : \forall (v_i, v_j) \notin E,~i \neq j,\\
\label{eq:mrcp-rest2}
&& \sum_{i=1}^{n}(a_ij)x_i ~ \geqslant 1 : \forall v_j \in V, \\
\label{eq:mrcp-rest3}
&& x_i \in {0,~1} : \forall v_i \in V.
\end{eqnarray}

El problema del clique de ratio máximo ha sido catalogado como un problema NP-difícil o NP-complejo, como se aprecia en la figura \ref{fig:np-dificil}, por lo que no es posible obtener una solución factible por métodos heurísticos o exactos.

\begin{figure}[H]
	\centering
	\includegraphics{Figures/problemas-np-hard.pdf}
	\caption{Diagrama de problemas Np y NP-difícil.}
	\label{fig:np-dificil}
\end{figure}

\subsection{Motivación del problema}

aaaaa

\section{Estado del arte}
El análisis sobre el estado del arte que se ha realizado sobre el problema del clique de ratio máximo y todo lo que rodea al mismo ha sido documentado mediante la siguiente literatura, si bien es cierto que no existe demasiada al respecto ya que se trata de un problema poco estudiado en comparación con el clásico problema de búsqueda de la clique máxima, \cite{mcp-batsyn}, \cite{mcp-ryp}, \cite{mcp-neuro}, o \cite{mcp-ants} donde se muestra un algoritmo \gls{ACO} con un optimizador local K-opt.

También se ha estudiado en mayor medida el problema de la búsqueda del clique de peso máximo, más cercano al tratado en este trabajo final de grado, como se puede observar en \cite{mwcp-ls} se implementan sendos algoritmos de búsqueda local llamados SCCWalk y SCCWalk4L, \cite{mwcp-ml} en el que se hace uso del muestreo estocástico y el machine-learning para abordar el problema eliminando variables de decisión que no formarían parte de una solución óptima.

En el documento desarrollado por Samyukta Sethuraman  y Sergiy Butenko \cite{mrcp-Sethuraman:2015} se trata el problema desde tres puntos de vista, mediante el IBM CPLEX para resolver el problema de manera lineal y la aplicación de la búsqueda binaria y el método de Newton.

En \cite{mrcp-moeni} se trata el problema basándose en el enfoque eficiente que dan las funciones de diferencia de convexos (DC) y sus algoritmos (DCA) el cuál provee resultados competitivos, a su vez, introducen las desigualdades válidas que ayudan a mejorar el tiempo computacional en la obtención de resultados de calidad al problema.

Cabe destacar entre todos los trabajos realizados sobre el problema del clique de ratio máximo el de Dominik Goeke, Mahdi Moeini y David Poganiuch \cite{mrcp-GOEKE2017283} el cuál se ha tomado como referencia para realizar este trabajo final de grado, en este artículo, los autores parten de un punto de vista multi-arranque \gls{MSM} añadiendo una búsqueda de vecindario variable \gls{VNS}, lo que permite obtener soluciones de gran calidad y un tiempo de computo menor.

%-------------------------------------------------------------------------------


% Chapter 2: Objectives

\chapter{Objetivos} % Main chapter title

\label{Chapter2}

%-------------------------------------------------------------------------------


En este capítulo se indican los objetivos, tanto principales como secundarios, establecidos al inicio de este trabajo fin de grado.

\section{Objetivos principales}
\begin{itemize}
	\item Estudio y comprensión del problema de la búsqueda del clique de ratio máximo.
	\item Estudio, diseño e implementación del algoritmo metaheurístico GRASP para la resolución del problema.
\end{itemize}

\section{Objetivos secundarios}
\begin{itemize}
	\item Conocer los distintos algoritmos metaheurísticos existentes.
	\item Comprender la complejidad computacional relacionada con la búsqueda del clique de ratio máximo.
	\item Aprendizaje del lenguaje de programación Python para el desarrollo del algoritmo metaheurístico GRASP.
	\item Profundización y mejora en técnicas algorítmicas, de programación y de estructuras de datos para la realización de este trabajo fin de grado.
\end{itemize}

%-------------------------------------------------------------------------------

% Chapter 3: Algorithm description

\chapter{Descripción algorítmica} % Main chapter title

\label{Chapter3}

%-------------------------------------------------------------------------------
En este capítulo se describe el algoritmo metaheurístico utilizado, exponiendo todas sus características para la obtención de una solución al problema.

\section{Metaheurística GRASP\index{GRASP}}
\label{sec_metaGrasp}
El acrónimo \gls{GRASP} (Greedy Randomized Adaptive Search Procedure) o en castellano procedimiento de búsqueda voraz aleatorizado y adaptativo, fue introducido por primera vez por Feo y Resende en 1995 en su artículo con el mismo nombre \cite{grasp-feo-resende}.

Este algoritmo se basa en el multi-arranque, dónde cada uno de ellos es una iteración de un procedimiento que está constituido por dos partes bien diferenciadas. Por un lado, la fase constructiva, en la que se obtiene una solución de buena calidad, y por otro una fase de mejora, en la que, partiendo de la solución obtenida en la fase anterior, se intenta mejorar localmente \cite{libro-metaheuristicas}. 
En \cite{grasp-flightrecoveryproblem} \cite{grasp-parallel} \cite{grasp-weapon} \cite{grasp-empaquetado} \cite{grasp-ruta} \cite{grasp-vertex} se pueden encontrar diversos documentos en los que se tratan problemas aplicando la metaheurística \gls{GRASP}.

En el algoritmo \ref{alg:grasp} se muestra el pseudocódigo de la metaheurística \gls{GRASP} que se ha empleado para el desarrollo y obtención de una solución preliminar para este problema, y posteriormente, se muestra el algoritmo \ref{alg:bl}, con el cual se ha refinado esta solución para obtener una mejor.\\

\subsection{Fase constructiva}
\label{sec:faseConstructiva}

\begin{algorithm}[H]
	\SetAlgoLined
	$ v \gets rnd( V ) $ \label{alg:grap:get_v} \\[0.2cm]
	$ S \gets \{ v \} $ \label{alg:grap:add_v_to_s} \\[0.2cm]
	$ CL \gets \{u \in V : (u, v) \in E\} $ \label{alg:grap:get_cl} \\[0.2cm]
	\While{$|CL| \not= 0$}{ \label{alg:grap:while} 
		$ \mathrm{g_{min}} \gets $ $ \smash{\displaystyle\min_{c \in CL}} \hspace{0.1cm} g(c) $ \label{alg:grap:get_gmax} \\[0.2cm]
		$ \mathrm{g_{max}} \gets $ $ \smash{\displaystyle\max_{c \in CL}} \hspace{0.1cm} g(c) $ \label{alg:grap:get_gmin} \\[0.2cm]
		$ \mu \gets  \mathrm{g_{max}} - \alpha ( \mathrm{g_{max}} - \mathrm{g_{min}} ) $ \label{alg:grap:get_mu} \\[0.2cm]
		$ RCL \gets \{ c \in CL : g(c) \geq \mu \}  $ \label{alg:grap:get_rcl} \\[0.2cm]
		$ u \gets rnd (RCL) $ \label{alg:grap:get_u} \\[0.2cm]
		$ S \gets S \cup \hspace{0.1cm} \{ u \}$ \label{alg:grap:add_u_to_s} \\[0.2cm]
		$ CL \gets CL \textbackslash \{ u \} \textbackslash \{ w : (u, w) \notin E \}$  \label{alg:grap:up_cl} \\[0.2cm]
	}
	\Return S \label{alg:grap:rt_s}
	\caption{Pseudocódigo de la fase constructiva del GRASP}
	\label{alg:grasp}
\end{algorithm}

Partiendo de un grafo $G=(V, E)$ donde $V$ son los vértices o nodos del grafo, y $E$ las aristas que unen estos nodos.\\
En primer lugar se toma un vértice $v$ aleatorio de entre los vértices del grafo y se incluye en la solución $S$ ya que cumple con las restricciones del problema, descritas en la sección \ref{intro-problema}. A partir de $v$ se construye la lista de candidatos $CL$, como se indica en el paso \ref{alg:grap:get_cl}, definida como todos los nodos adyacentes a $v$ que forman parte de la lista de nodos del grafo de partida. A continuación, se toma un elemento de la lista de candidatos y se obtiene mediante una función voraz un listado de valores para la restricción posterior de la lista de candidatos.
La función voraz será determinada antes de iniciar el proceso y puede ser:

\begin{itemize}
	\item Función voraz por ratio: Obtiene el listado de vecinos prometedores basando su selección de mejor nodo en el valor obtenido entre los pesos $p$ y $q$ asociados a este. El valor es calculado como $\frac{p_i}{q_i}$, donde $i$ es el nodo tratado.
	\item Función voraz por número de adyacentes: En este caso, se selecciona como mejor nodo el que más vecinos tenga, esto se denota como $max$ $\omega_i$, siendo $\omega$ la cardinalidad del nodo e $i$ el nodo tratado.
\end{itemize}

De este listado de valores de ratio se escogen el máximo y mínimo, descrito en los pasos \ref{alg:grap:get_gmax} y \ref{alg:grap:get_gmin}, y junto a $\alpha$ se obtiene el valor de $\mu$, mostrado en el paso \ref{alg:grap:get_mu}.

El valor de $\mu$ indica el umbral de la lista restringida de candidatos $RCL$. Esta lista se genera mediante la lista de candidatos $CL$ ordenada de mayor a menor valor y, como se muestra en la figura \ref{fig:rcl}, se toman los primeros valores hasta alcanzar el valor umbral, esa función está descrita  en el paso \ref{alg:grap:get_rcl}.

 \begin{figure}[H]
	\centering
	\includegraphics{Figures/rcl.pdf}
	\caption{Detalle de la lista restringida de candidatos (RCL).}
	\label{fig:rcl}
\end{figure}

Respecto al valor de $\alpha$, será configurado antes de este proceso y oscilará entre $0$ y $1$. Este valor va a determinar la cantidad de nodos que se incluirían en la $RCL$. Por un lado, valores cercanos a $1$ van a incluir un mayor número de nodos en la lista, mientras que valores cercanos a $0$ harán que el algoritmo se comporte de una manera más determinista, incluyendo menos valores en el listado de candidatos.

De la $RCL$ se elegirá, de manera aleatoria, un nodo $u$ como se muestra en el paso \ref{alg:grap:get_u}. Este se añadirá a la lista solución $S$, paso \ref{alg:grap:add_u_to_s} y será eliminado de la lista de candidatos junto con los nodos que no sean adyacentes a este, paso \ref{alg:grap:up_cl}.

Este procedimiento será repetido hasta que la lista de candidatos este vacía, obteniéndose en ese momento la lista $S$ final, que conformará la solución preliminar y será retornada por la función.

A continuación, se muestra un ejemplo de adición de un nuevo nodo al clique solución:

Partiendo de la situación que se muestra en la figura \ref{fig:const:cliq-cl}, donde  $S = \{v_A, v_B\}$ es una solución parcial y $ CL = \{C,D,E\} $ es la lista de candidatos actual.

\begin{figure}[H]
	\centering
	\includegraphics[scale=2]{Figures/proc-const/clique-y-cl.pdf}
	\caption{\footnotesize Ejemplo de solución parcial y lista de candidatos actual.}
	\label{fig:const:cliq-cl}
\end{figure}

Mediante la función voraz seleccionada en la configuración inicial, se genera un listado de candidatos factibles y prometedores, indicado en la figura \ref{fig:const:func-voraz}, a partir de los nodos de la actual $CL$.

Cada posición corresponde con el nodo de partida de $CL$ y tiene asociado el valor del ratio calculado de los nodos obtenidos mediante la función voraz: $f(r_C)=0.76$ , $f(r_D)=2$ y $f(r_E)=1.44$.

\begin{figure}[H]
	\centering
	\includegraphics[scale=2]{Figures/proc-const/func-voraz.pdf}
	\caption{\footnotesize Ejemplo de obtención del listado de valores mediante la función voraz.}
	\label{fig:const:func-voraz}
\end{figure}

De este listado de valores se obtienen el valor máximo $\mathrm{g_{max}}=2$, y el mínimo $\mathrm{g_{min}}=0.76$. Con estos y el valor de $\alpha = 0.5$ por ejemplo, se calculará el valor de $\mu$ de la siguiente manera:
\begin{center}
	$\mu = 2 - 0.5 * (2 - 0.76) = 1.38$
\end{center}

Con este valor, se marca el umbral de selección para la lista restringida de candidatos o $RCL$, como se indica en la figura \ref{fig:const:rlc}, donde se incluirían solamente los nodos $D$ y $E$. De esta lista será seleccionado aleatoriamente un nodo, por ejemplo $D$, que será añadido a la solución parcial, quedando como $S = \{A, B, D\}$. Acto seguido, se eliminarán de $CL$ todos los nodos que no sean adyacentes a $D$ y se repetirá este proceso descrito hasta que $CL$ quede vacía.

\begin{figure}[H]
	\centering
	\includegraphics[scale=2]{Figures/proc-const/rcl.pdf}
	\caption{\footnotesize Ejemplo de obtención de la lista de candidatos restringida.}
	\label{fig:const:rlc}
\end{figure}

%
%\begin{figure}[H]
%	\hspace{-1.5cm}
%	\begin{minipage}[b]{0.55\linewidth}
%		\centering
%		\includegraphics[scale=1.6]{Figures/diag-const-ratio.pdf}
%		\caption{\footnotesize Elección de nodo por mayor ratio.}
%		\label{fig:const-ratio}
%	\end{minipage}
%	\hspace{-1cm}
%	\begin{minipage}[b]{0.6\linewidth}
%		\centering
%		\includegraphics[scale=1.6]{Figures/diag-const-ady.pdf}
%		\caption{\footnotesize Elección de nodo por mayor número de adyacentes.}
%		\label{fig:const-ady}
%	\end{minipage}
%\end{figure}
%En el caso de la figura \ref{fig:const-ratio} se escoge el nodo $D$ que tiene una relación mayor entre sus pesos $p$ y $q$, $f(v_D)=\frac{58}{29}=2$, respecto del nodo $E$. Como siempre se selecciona un solo nodo en el paso \ref{alg:const_voraz:get_mejor} del algoritmo \ref{alg:const_voraz}, en lugar de un subconjunto de nodos, este caso puede suponer un mayor ratio final.
%
%Por otro lado, en la figura \ref{fig:const-ady}, se observa la opción de escoger el nodo con mayor número de vecinos. A priori, puede suponer una mejor solución añadir mayor cantidad de adyacentes y, por lo tanto, mayor ratio. Al tratarse de un algoritmo voraz, esta estrategia puede dejar atrás nodos más prometedores para alcanzar mayor ratio en la solución final. En este ejemplo, se escoge como mejor nodo el $E$, ya que tiene más adyacentes. Posteriormente se añade, por ejemplo, el nodo $G$, obteniendo un ratio de $f(\{v_A,v_B,v_C,v_E,v_G\})=0,96$. Esta selección está dejando atrás los nodos $D$ e $I$, los cuáles aportan mayor ratio a la solución, con un valor en este caso de $f(\{v_A,v_B,v_C,v_D,v_I\})=1,46$, superior al obtenido mediante el otro constructivo.

\subsection{Fase de mejora}
\label{sec:faseBusqueda}
Para esta segunda fase, se ha definido el algoritmo \ref{alg:bl}, el cuál parte de la solución obtenida previamente en la fase constructiva, formada por los nodos que forman un clique.

\begin{algorithm}
	$ soluciones \gets [] $ \\[0.2cm]
	$ vecinos \gets \emptyset $ \\[0.2cm] \label{alg:mj:vecVacios}
	$ vecinos \gets obtenerVecinos(solucion)$ \\[0.2cm] \label{alg:mj:getVecSol}
	$ vecinosOrdenados \gets ordenarVecinos(vecinos)$ \\[0.2cm] \label{alg:mj:ordVecs}
	\For{nodo $\epsilon$ vecinosOrdenados}{ \label{alg:mj:for}
		$ solucion\_parcial \gets incluirNodo(solucion, nodo) $ \\[0.2cm] \label{alg:mj:addNodoSol}
		\If{$ no \hspace{0.1cm} esClique(solucion) $}{ \label{alg:mj:conNoClique}
			$ solucion\_parcial \gets excluirNodos(solucion, nodo) $ \\[0.2cm] \label{alg:mj:exNodo}
		}\Else{
			$ solucion\_parcial \gets incluirAdyacentes(solucion\_parcial, nodo) $ \\[0.2cm] \label{alg:mj:addAdys}
			$ soluciones \gets incluirSolucion(solucion\_parcial) $
			 \\[0.2cm] \label{alg:mj:inclSolParc} 
		}
	}
	$ solucion \gets seleccionarMejorSolucion(soluciones) $
	\\[0.2cm] \label{alg:mj:selMejorSol}
	\Return solucion \label{alg:mj:rtSol}
	\caption{Pseudocódigo del algoritmo de búsqueda local.}
	\label{alg:bl}
\end{algorithm}

Con esta solución se obtienen en el paso \ref{alg:mj:getVecSol} todos los vecinos de cada nodo. Estos son ordenados de mayor a menor ratio en el paso \ref{alg:mj:ordVecs}.\\
Siguiendo con el ejemplo de la fase anterior nos quedaría un listado de nodos vecinos. Esta ordenación se realiza con el fin de aumentar las posibilidades de obtener un mejor valor de ratio. 

Se obtiene el primer nodo del listado y se añade a la solución en el paso \ref{alg:mj:addNodoSol}, comprobando posteriormente si esta solución forma o no un nuevo clique en el paso \ref{alg:mj:conNoClique}. Esta comprobación verifica que todos los nodos son adyacentes entre sí, formando un clique.

Si añadir el nodo no formara una solución, en el paso \ref{alg:mj:exNodo} se excluyen todos los nodos que impiden que se forme una solución factible.\\
En caso contrario, ese nodo añadido en el paso \ref{alg:mj:addNodoSol} se mantendría y a su vez, en el paso \ref{alg:mj:addAdys}, se añadirían todos sus nodos adyacentes, con el fin de obtener un clique de tamaño máximo, como marca la restricción del problema. El criterio de selección de los adyacentes es al igual que en la fase anterior, mediante una función voraz, en este caso por mayor ratio de entre los adyacentes disponibles.


Esa solución parcial es añadida al listado soluciones en el paso \ref{alg:mj:inclSolParc}. Tras procesarse todos los nodos, se seleccionará la solución con mayor ratio calculado y será retornada en el paso \ref{alg:mj:rtSol}.

A continuación, al igual que en la fase anterior se muestra un ejemplo del proceso.

Partiendo de una posible solución previa $ S=\{A, B, C, D, E\} $, en la figura \ref{fig:bl:adys} se muestra la lista ordenada por ratio de todos los nodos adyacentes de la solución previa. 

\begin{figure}[H]
	\centering
	\includegraphics[scale=2]{Figures/proc-bl/adys-ord.pdf}
	\caption{\footnotesize Lista de adyacentes ordenada.}
	\label{fig:bl:adys}
\end{figure}

De este listado se selecciona la primera posición correspondiente en este caso con el nodo $A_1$ ya que tiene el mayor ratio y es añadido a la solución parcial $S_{A_1}$.

A continuación se comprueba si esta nueva solución cumple con las restricciones para ser clique. Si no cumple con las restricciones se aplicaría el proceso de exclusión de nodos que evitan que se forme un clique. En la imagen \ref{fig:bl:exc} se muestra el proceso, en el que los nodos sombreados $B$ y $C$ son eliminados de la solución.

\begin{figure}[H]
	\centering
	\includegraphics[scale=1.3]{Figures/proc-bl/excluirNodos.pdf}
	\caption{\footnotesize Exclusión de los nodos que impiden obtener un clique factible.}
	\label{fig:bl:exc}
\end{figure}

Una vez se tiene una solución parcial factible, se añaden todos los nodos adyacentes a los de la solución para obtener un clique máximo. En la figura \ref{fig:bl:add-ayd} se muestra el proceso de adición de los adyacentes mediante una función voraz, en este caso el criterio de selección es por el que aporte mayor ratio a esa solución parcial. Como se indicó anteriormente este ratio es calculado con los pesos $p$ y $q$ asociados al nodo en cuestión.

 Esta será añadida al listado de soluciones parciales.

\begin{figure}[H]
	\centering
	\includegraphics[scale=1.3]{Figures/proc-bl/add-adys.pdf}
	\caption{\footnotesize Adición de adyacentes a la solución parcial.}
	\label{fig:bl:add-ayd}
\end{figure}

Este proceso será repetido con todos los nodos de la lista ordenada generada al inicio y finalmente se devolverá la solución que ha obtenido mayor valor de ratio calculado.

%-------------------------------------------------------------------------------
% Chapter 4: Design and Implementation

\chapter{Descripción informática} % Main chapter title

\label{Chapter4} % Reference

%-------------------------------------------------------------------------------

En este capítulo se describe el desarrollo completo del proyecto, desde el diseño  hasta la implementación de este, así como la metodología utilizada para la correcta evolución del mismo.

\section{Diseño}
aaa

\section{Implementación}
La implementación de este proyecto se basa en la ejecución de un script\footnote{script} escrito en lenguaje Python el cual parte de una clase en la que se encuentra la siguiente sentencia de código:
 \begin{lstlisting}[language=Python]
 if__name__ == "__main__"
 \end{lstlisting}
 la cual posibilita la ejecución del script como aplicacion standalone\footnote{standalone} mediante el comando:
  \begin{lstlisting}[language=bash]
  python grasp_main.py
 \end{lstlisting}
 Este script, en adelante Grasp Main, tiene la configuración necesaria para ajustar el programa:
 \begin{itemize}
 	\item Número de iteraciones a realizar por cada fichero.
 	\item Ruta donde se encuentran los ficheros de definición de los grafos a procesar.
 	\item Ruta de los ficheros de resultados generados por el programa.	
 \end{itemize}

Grasp Main se encarga de recorrer recursivamente los ficheros que se encuentran en la ruta de recursos definida y por cada uno de los ficheros encontrados crea un objeto de tipo Instance, añadiendo la información del grafo:
 \begin{itemize}
	\item Número de nodos.
	\item Número de aristas.
	\item Estructura de datos con los nodos del grafo.
\end{itemize}

Esta estructura de datos contiene tantos objetos de tipo Node como tengo el grafo, cada uno de ellos con la siguiente información:
 \begin{itemize}
	\item Identificador del nodo.
	\item Valor del peso p.
	\item Valor de peso q.
	\item Grado del nodo.
	\item Estructura de datos con las relaciones de este nodo con otros nodos del grafo.
\end{itemize}

Tras terminar esta operación, realiza el procesado un número N de veces, según se haya definido previamente en la configuración de la aplicación en Grasp Main, y por cada tipo de constructivo de los que dispone la aplicación, en este caso, constructivo ratio y constructivo adyacente, los cuales serán explicados más adelante. 
La implementación del algoritmo GRASP se encuentra en la clase SolutionGrasp y contiene los métodos necesarios para la obtención mediante el algoritmo de una solución, partiendo de la función $find\_grasp\_solutio$n, la cual inicializa los siguientes datos:
\begin{itemize}
	\item vertex, el cuál es obtenido de manera aleatoria entre todos los nodos del grafo.
	\item solution, conjunto inicializado con el vértice obtenido anteriormente.
	\item cl, lista de candidatos posibles para encontrar una solución.
\end{itemize}

//TODO ¿se debe explicar cada método que forma parte de esta funcion?

Esta función se apoya en otra auxiliar, nombrada como $ find\_solution\_aux $, la cuál implementa el algoritmo utilizado, de esta manera se desacopla el tipo de algoritmo de la búsqueda de una solución, dando la posibilidad a un cambio posterior si fuera necesario. Dicha función, procesará los nodos del grafo tal como se describió en el capítulo anterior. Para la fase constructiva del algoritmo, según el tipo elegido en la configuración inicial, se creará un objeto del contructivo específico, SolutionGreedyRatio o SolutionGreedyAdjacent. Estos heredan de la clase de la clase abstracta\footnote{clase abstracta} SolutionGreedy, la cuál tiene la información compartida por ambos tipos, y delega la implementación de la función $find\_better$ en cada una de las clases específicas, quienes mediante un algoritmo de tipo voráz buscan una solución factible en un tiempo muy reducido.

Para mantener cierta información en un único lugar se ha implementado la clase GraphUtils, la cuál contiene información necesaria para los grafos y métodos útiles para usarse durante el procesado de los ficheros.

Para probar las diferentes funciones del proyecto se han escrito casos de prueba mediante la librería de Python unittest\footnote{unittest}, para conseguir un código tolerable a posibles fallos y mantenible.

//TODO Explicación del proceso, partiendo de la lectura de fichero, luego crear solucion voraz


\section{Metadología empleada}
Para el desarrollo de este proyecto se ha optado por seguir una metodología de tipo iterativa e incremental, lo que permite evolucionar el proyecto progresivamente e ir adaptando los requisitos del cliente, en este caso los tutores del proyecto, en el menor tiempo posible, mejorando así la calidad del producto final con el menor esfuerzo.

Estas iteraciones e incrementos de funcionalidad se han realizado durante todo el desarrollo del proyecto, mediante reuniones, con un lapso de aproximadamente 3 a 4 semanas entre ellas, corrigiendo errores de la iteración anterior si los hubiera y aumentando la funcionalidad del producto a entregar tanto en funcionalidad como en calidad, de esta manera se consigue una evolución progresiva y segura sobre el producto y los requerimientos que el proyecto exige.

Para mantener el control y administrar lo correspondiente sobre las tareas a realizar, las que se han realizado y las realizadas del proyecto, se ha usado la utilidad Trello\footnote{https://trello.com/es}, la cuál mediante tarjetas sobre el tablero del proyecto permite conocer las tareas del proyecto, así como conocer su estado, y añadir nuevas si así fuera necesario.

En cuanto al mantenimiento de versiones del proyecto se ha usado el sistema de control de versiones Git\footnote{https://git-scm.com/} a través del portal de alojamiento de repositorios GitHub\footnote{https://github.com/}, para interactuar entre el repositorio local y el repositorio remoto se ha optado por hacer uso tanto de la terminal mediante los comandos del propio sistema de control de versiones Git como del cliente para tal propósito GitKraken\footnote{https://www.gitkraken.com/}, el cuál permite mediante su sencilla e intuitiva interfaz mantener un control exhaustivo sobre las ramas y versionado de las distintas piezas de código del proyecto en el que se trabaja, así como revisar posibles conflictos que se produzcan.
%-------------------------------------------------------------------------------


% Chapter 5: Results

\chapter{Resultados} % Main chapter title

\label{Chapter5} % Reference

%-------------------------------------------------------------------------------

En este capítulo se exponen los diferentes recursos, tanto hardware como software, para la realización de este trabajo fin de grado y cómo, a partir de estos, se han obtenido los resultados para su posterior análisis.

\section{Recursos utilizados}
A continuación se detallará la máquina y software empleados para el desarrollo del código, así como las instancias utilizadas para comprobar la calidad del algoritmo .

\subsection{Descripción de la máquina utilizada}
\label{sec:maquina}
Para la realización de las diversas pruebas y procesado de las instancias de este problema se ha utilizado una máquina con las siguientes características:

\begin{itemize}

\item \textbf{Procesador:} Intel(R) Core(TM) i5-5257U CPU 2.70 GHz
\item \textbf{Memoria RAM:} 8 GB 1867 Mhz DDR3
\end{itemize}

El desarrollo del código se ha realizado mediante el lenguaje de programación Python\footnote{https://www.python.org/}, en su versión 3.7.4, a través del entorno de desarrollo integrado o \gls{IDE} (Integrated Development Environment) PyCharm\footnote{https://www.jetbrains.com/es-es/pycharm/} de JetBrains en su versión 2019.3.1.

\subsection{Instancias utilizadas}
\label{sec:Instancias-utilizadas}
Las instancias con las que se ha contado para comprobar la eficiencia del algoritmo desarrollado, han sido proporcionadas por los estudios previos en los que se basa y compara este trabajo final de grado. Estas hacen referencia a conjuntos de grafos, tanto generados de manera aleatoria como obtenidos de diferentes fuentes de datos, como precios del mercado de valores o turbinas de viento.

A continuación se detallan los diferentes conjuntos en los que se dividen las instancias:

\begin{itemize}
	
	\item \textbf{Conjuntos de tipo A y B:} Estos conjuntos son instancias de grafos generadas mediante una distribución de probabilidad uniforme, variando su número de nodos entre 100 y 500, la densidad de estos grafos oscila entre el 45,78 \% y el 53,64 \%.
	\item  \textbf{Conjunto de tipo C:} Los datos pertenecientes a las instancias de este tipo hacen referencia a datos de los precios del mercado de valores.
	\item  \textbf{Conjunto de tipo D:} Las instancias de este conjunto son datos para la construcción de turbinas de viento, donde cada nodo representa una localización de estas turbinas y sus pesos son la media de la velocidad del viento y el coste de contrucción de una turbina en ese punto.
	\item  \textbf{Conjunto de tipo E:} Estas instancias están extraídas del segundo y décimo DIMACS Implementation Challenge\footnote{http://dimacs.rutgers.edu/programs/challenge/}, donde cada nodo tiene un peso $\mathrm{p_{i} = 1}$ y un peso $\mathrm{q_{i} = 2}$. Adicionalmente se ha añadido un nodo más a cada instancia de este conjunto, el cual está conectado al resto de nodos de la misma, con un peso $\mathrm{p_{i} = 1}$ y un peso $\mathrm{q_{i} = 1}$, donde i es el número del nodo dentro de esa instancia.
	\item  \textbf{Conjunto de tipo F:} Estas instancias son las mismas que en el conjunto E pero en este caso los pesos de cada nodo son $\mathrm{p_{i} = i}$ y $\mathrm{q_{i} = |V| - i + 1}$, donde i es el número del nodo dentro de esa instancia.
	
	
\end{itemize}

%-------------------------------------------------------------------------------

\section{Análisis de los resultados}
En esta sección se muestran los resultados obtenidos tras el procesamiento de las instancias descritas en la sección \ref{sec:Instancias-utilizadas} y empleando la máquina descrita en la sección \ref{sec:maquina}. Estos resultados se han obtenido en dos fases, una primera fase preliminar, con un conjunto reducido de las instancias y una segunda fase, con todas las instancias del problema para su posterior comparación con los resultados de estudios previos.

\subsection{Experimentos preliminares}
En esta fase ha sido seleccionado un subconjunto de 22 instancias del problema, con el fin de estudiar la calidad del algoritmo y comprobar los resultados obtenidos mediante distintos valores para $\alpha$ en varias iteraciones, estas se han elegido de todos los conjuntos para ampliar el rango de las pruebas. El número de nodos en estas instancias varia desde los 30 hasta los 1000 nodos, con densidades del 1 al 50 por ciento aproximadamente.

A esta selección de instancias se les ha aplicado los dos constructivos desarrollados, descritos en la sección \ref{sec:implementacion}, y a su vez se han aplicado los valores de $\alpha$ $\epsilon$ \{0.25, 0.5, 0.75\} y adicionalmente un valor de $\alpha$ aleatorio.
Este experimento se ha realizado procesando cada instancia 100 veces, que unido a lo anteriormente descrito, hace un total de 800 iteraciones por fichero.

Durante los experimentos preliminares se han recopilado los datos mostrados en la tabla \ref{table:pre-ady} para el constructivo por número de adyacentes y la tabla \ref{table:pre-ratio} para el constructivo por el valor del ratio. Las tablas están divididas en el uso del algoritmo \gls{GRASP} y su posterior mejora mediante la búsqueda local y donde:

\begin{itemize}
	\item $f$: Indica el valor de la función objetivo calculado.
	\item c: Indica la cardinalidad del clique encontrado.
	\item t: Indica el tiempo total empleado en encontrar la solución en segundos.
	\item $\alpha$ : Indica el valor de $\alpha$ con el que se ha obtenido esa solución.
	\item Mejor: Indica si la solución obtenida mediante este constructivo es mejor respecto al otro.
\end{itemize}

\begin{scriptsize}
\pgfplotstabletypeset[
multicolumn names,
empty header,
begin table=\begin{longtable},
	every first row/.append style={before row={
			\caption{Resultados de los experimentos preliminares con constructivo adyacentes.}
			\label{table:pre-ady}\\\toprule
			\multicolumn{1}{c|}{\textbf{Instancias}} &
			\multicolumn{4}{c}{\textbf{GRASP}} &
			\multicolumn{5}{|c}{\textbf{GRASP y búsqueda local}} \\
			\multicolumn{1}{c|}{} & \multicolumn{1}{c}{\textbf{$f$}} & \textbf{c} &\textbf{t (sec)} & \textbf{\hspace{0.5cm}$\alpha$} & \multicolumn{1}{|c}{\textbf{$f$}} & \textbf{c} & \textbf{t (sec)} & \textbf{\hspace{0.5cm}$\alpha$} & \textbf{Mejor} \\ \toprule    
			\endfirsthead
			%
			\multicolumn{10}{c}
			{{\bfseries Tabla \thetable\ Continuación de la página anterior}} \\
			\toprule 
			\multicolumn{1}{c|}{\textbf{Instancias}} &
			\multicolumn{4}{c}{\textbf{GRASP}} &
			\multicolumn{5}{|c}{\textbf{GRASP y búsqueda local}} \\
			\multicolumn{1}{c|}{} & \multicolumn{1}{c}{\textbf{$f$}} & \textbf{c} &\textbf{t (sec)} & \textbf{\hspace{0.5cm}$\alpha$} & \multicolumn{1}{|c}{\textbf{$f$}} & \textbf{c} & \textbf{t (sec)} & \textbf{\hspace{0.5cm}$\alpha$} & \textbf{Mejor} \\ \toprule    
			\endhead
			\midrule \multicolumn{10}{r}{{Continúa en la siguiente página}} \\ \bottomrule
			\endfoot
			\midrule
			\multicolumn{10}{r}{{Concluido}} \\ \bottomrule
			\endlastfoot
	}},
	end table=\end{longtable},
col sep=semicolon,
string type,
display columns/0/.style={postproc cell content/.append style={@cell content={\textbf{##1}}}},
display columns/1/.style={column type={S}},
display columns/2/.style={column type={c}},
display columns/3/.style={column type={S}},
display columns/4/.style={column type={S}},
display columns/5/.style={column type={S}},
display columns/6/.style={column type={S}},
display columns/7/.style={column type={S}},
display columns/8/.style={column type={S}},
display columns/9/.style={column type={c}},
]{Results/pre-ady-fin.csv}

\pgfplotstabletypeset[
multicolumn names,
empty header,
begin table=\begin{longtable},
	every first row/.append style={before row={
			\caption{Resultados de los experimentos preliminares con constructivo ratio.}
			\label{table:pre-ratio}\\\toprule
			\multicolumn{1}{c|}{\textbf{Instancias}} &
			\multicolumn{4}{c}{\textbf{GRASP}} &
			\multicolumn{5}{|c}{\textbf{GRASP y búsqueda local}} \\
			\multicolumn{1}{c|}{} & \multicolumn{1}{c}{\textbf{$f$}} & \textbf{c} &\textbf{t (sec)} & \textbf{\hspace{0.5cm}$\alpha$} & \multicolumn{1}{|c}{\textbf{$f$}} & \textbf{c} & \textbf{t (sec)} & \textbf{\hspace{0.5cm}$\alpha$} & \textbf{Mejor} \\ \toprule    
			\endfirsthead
			%
			\multicolumn{10}{c}
			{{\bfseries Tabla \thetable\ Continuación de la página anterior}} \\
			\toprule 
			\multicolumn{1}{c|}{\textbf{Instancias}} &
			\multicolumn{4}{c}{\textbf{GRASP}} &
			\multicolumn{5}{|c}{\textbf{GRASP y búsqueda local}} \\
			\multicolumn{1}{c|}{} & \multicolumn{1}{c}{\textbf{$f$}} & \textbf{c} &\textbf{t (sec)} & \textbf{\hspace{0.5cm}$\alpha$} & \multicolumn{1}{|c}{\textbf{$f$}} & \textbf{c} & \textbf{t (sec)} & \textbf{\hspace{0.5cm}$\alpha$} & \textbf{Mejor} \\ \toprule    
			\endhead
			\midrule \multicolumn{10}{r}{{Continúa en la siguiente página}} \\ \bottomrule
			\endfoot
			\midrule
			\endlastfoot
	}},
	end table=\end{longtable},
col sep=semicolon,
string type,
display columns/0/.style={postproc cell content/.append style={@cell content={\textbf{##1}}}},
display columns/1/.style={column type={S}},
display columns/2/.style={column type={c}},
display columns/3/.style={column type={S}},
display columns/4/.style={column type={S}},
display columns/5/.style={column type={S}},
display columns/6/.style={column type={S}},
display columns/7/.style={column type={S}},
display columns/8/.style={column type={S}},
display columns/9/.style={column type={c}},
]{Results/pre-ratio-fin.csv}
\end{scriptsize}

En estas tablas se encuentran los mejores resultados tras todas las iteraciones de ambos constructivos y la posterior mejora del algoritmo con la búsqueda local.

Aunque en algunos casos, añadir al algoritmo la mejora mediante la búsqueda local no parece aumentar la calidad de la solución, en la mayoría de casos si, por lo que finalmente se añadirá para la experimentación final. 

Además, tras estas iteraciones se ha comprobado que el constructivo mediante el cálculo por ratio ha obtenido mejores resultados en relación ratio-cardinalidad, alcanzado 13 soluciones mejores respecto a las 9 soluciones mejores obtenidos mediante el constructivo por adyacentes. 

Por lo tanto el constructivo basado en el ratio, permite obtener valores para la función objetivo mayores y cliques con mayor número de nodos en tiempos de computo razonables. A esto hay que añadir que el valor de $\alpha$ con el que se han obtenido mejores resultados ha sido 0.75, llegando a 16 soluciones mejores de las 22 instancias procesadas en este caso.


\subsection{Experimento final}

En esta sección se han recopilado todos los datos obtenidos en las pruebas preliminares para el procesado final y se ha configurado de la siguiente manera:
\begin{itemize}
	\item Algoritmo \gls{GRASP}.
	\item Búsqueda local.
	\item Constructivo calculado por ratio.
	\item Valor de $\alpha = 0.75$. 
	\item 100 iteraciones por instancia.
\end{itemize}

Con los que tras el procesado de todas las instancias, se han obtenido los datos mostrados en la tabla \ref{table:exp-final}, donde:
\begin{itemize}
	\item $f$: Indica el valor de la función objetivo calculado.
	\item c: Indica la cardinalidad del clique encontrado.
	\item t: Indica el tiempo total empleado en encontrar la solución en segundos.
	\item $\Delta f (\%)$: Indica la variación respecto al valor de la función objetivo en la que se ha basado este trabajo.
	\item Mejor: Indica si la solución obtenida mediante el algoritmo es mejor o no que la del trabajo anterior.
\end{itemize}

Los datos recopilados durante el experimento final han sido comparados con los obtenidos previamente en el trabajo realizado por Dominik Goeke, Mahdi Moeini y David Poganiuch \cite{mrcp-GOEKE2017283}, los cuales se pueden observar en la tabla 2 del mismo. 

\begin{footnotesize}
\pgfplotstabletypeset[
empty header,
begin table=\begin{longtable},
	every first row/.append style={before row={%
			\caption{Resultados del experimento final.}
			\label{table:exp-final}\\\toprule
			\textbf{Instancias} &\textbf{$f$} &\textbf{c} &\textbf{t (sec)} &\textbf{$\Delta f(\%)$} &\textbf{$Mejor$} \\ \toprule    
			\endfirsthead
			\multicolumn{6}{c}%
			{{\bfseries Tabla \thetable\ Continuación de la página anterior}} \\
			\toprule 
			\textbf{Instancias} &\textbf{$f$} &\textbf{c} &\textbf{t (sec)} &\textbf{$\Delta f(\%)$} &\textbf{$Mejor$} \\ \toprule    
			\endhead
			\midrule \multicolumn{6}{r}{{Continúa en la siguiente página}} \\ \bottomrule
			\endfoot
			\midrule
			\endlastfoot
	}},
	end table=\end{longtable},
col sep=semicolon,
string type,
display columns/0/.style={postproc cell content/.append style={@cell content={\textbf{##1}}}},
display columns/1/.style={column type={S}},
display columns/3/.style={column type={S}},
]{Results/tabla-final.csv}
\end{footnotesize}

Como indica el porcentaje de variación de la función objetivo, $\Delta f (\%)$, se puede observar que en todas las instancias se han obtenido valores, en algunos casos, muy cercanos al trabajo anterior, ya que son valores cercanos a 0, y por lo tanto el algoritmo cumple cierta calidad. Aún así, sólo en 10 casos se han obtenido valores mejores.

También cabe destacar que los resultados obtenidos se han realizado en un tiempo de computo significativamente bajo, por lo que aunque no supere al trabajo de referencia, si mejora los resultados obtenidos mediante la linealización y el método de Newton con el solucionador Gurobi\footnote{https://www.gurobi.com/es/} con los que se compara dicho trabajo.

%-------------------------------------------------------------------------------










% Chapter 6: Last Conclusions

\chapter{Conclusiones} % Main chapter title

\label{Chapter6} % Reference

%-------------------------------------------------------------------------------

En este capítulo se describen las conclusiones finales alcanzadas tras el desarrollo del proyecto, así como las lecciones aprendidas durante el mismo.

\section{Consecución de los objetivos}

Los objetivos establecidos al comienzo del proyecto son:

\begin{itemize}
	\item Estudio y comprensión del problema de la búsqueda del clique de ratio máximo.
	\item Estudio e implementación del algoritmo metaheurístico \gls{GRASP} para la resolución del problema.
\end{itemize}

Y por otro lado los objetivos secundarios:
\begin{itemize}
	\item Estudio de la importancia de los grafos en la vida real.
	\item Estudio de los distintos algoritmos metaheurísticos existentes.
	\item Estudio y comprensión de la complejidad computacional relacionada con la búsqueda de clique de ratio máximo.
	\item Aprendizaje del lenguaje de programación Python para el desarrollo del algoritmo metaheurístico \gls{GRASP}.
	\item Profundización y mejora en técnicas algorítmicas, de programación y de estructuras de datos para la realización de este trabajo fin de grado.
\end{itemize}

Estos objetivos se han cumplido de manera satisfactoria, puesto que se ha estudiado el estado del arte del problema del clique de ratio máximo para poder abordarlo, así como problemas relacionados como son el problema del clique de peso máximo (\gls{MWCP}) y el clásico problema del clique máximo (\gls{MCP}).
También se han estudiado las distintas familias en las que se categorizan los algoritmos metaheurísticos y en especial el algoritmo \gls{GRASP}.
Todo esto ha ayudado en la comprensión del problema y su posterior diseño e implementación.

%-------------------------------------------------------------------------------

\section{Conocimientos adquiridos}

El proceso de realización de este trabajo fin de grado ha supuesto la superación de diferentes retos los cuales han permitido ampliar conocimientos sobre el desarrollo de software y la gestión de un proyecto. También se ha profundizado en conceptos sobre algoritmia y estructuras de datos para obtener soluciones mejores y más eficientes al planteamiento de problemas de optimización. Destacando:

\begin{itemize}
  \item El aumento y mejora de conocimientos en el lenguaje de programación Python, usado para la implementación de este proyecto.
  \item La comprensión sobre conceptos de algoritmia y estructuras de datos, así como la mejora continua del código, enfocado en la resolución de problemas de optimización.
  \item Adquisición de conocimientos sobre heurísticas y metaheurísticas aplicadas a la resolución de problemas.
  \item Profundización en el uso de grafos en programación, así como la relación con el mundo real.
  \item El aprendizaje sobre \LaTeX, al utilizarlo para documentar el trabajo fin de grado.
\end{itemize}

\section{Líneas de desarrollo futuras}
En cuanto a las líneas de desarrollo futuras para este problema se describen algunas ideas como:

En primer lugar, un caso que ofrece un considerable aumento en el rendimiento del código, es la utilización de Cython\footnote{https://cython.org/} mediante pequeñas modificaciones en el código para ajustarlo a su lenguaje propio. Este compilador optimizado permite aumentar el  rendimiento de las funciones escritas en Python. Algunas de sus características son la unificación de la legibilidad del código escrito en Python con el rendimiento de C, y la interacción eficiente con grandes conjuntos de datos mediante NumPy\footnote{https://numpy.org/}.

Otra opción podría ser la implementación del procesado paralelo aprovechando el módulo $concurrent.futures$ que ofrece Python, este hace uso del módulo multiprocessing, por lo tanto no se ve afectado por el \gls{GIL}\footnote{https://docs.python.org/3/glossary.html?highlight=gil\#term-global-interpreter-lock} a diferencia del módulo multithreading, y permite crear un conjunto de procesos, mediante la función ProcessPoolExecutor, para ejecutar llamadas asíncronas. 

Esta opción se puede añadir para el cálculo de los valores a partir de la lista de todos los nodos candidatos, como se describió en la sección \ref{sec_metaGrasp} en el algoritmo \ref{alg:grasp}, con el fin de obtener los resultados de manera paralela, reduciendo considerablemente el tiempo de cómputo, ya que para instancias con una gran cantidad de nodos y una alta densidad este tiempo puede ser excesivo.

%-------------------------------------------------------------------------------


%-------------------------------------------------------------------------------
%	THESIS CONTENT - APPENDICES
%-------------------------------------------------------------------------------

%\appendix % Cue to tell LaTeX that the following "chapters" are Appendices

% Include the appendices of the thesis as separate files from the Appendices folder
% Uncomment the lines as you write the Appendices

%% Appendix A

\chapter{Anexos} % Main appendix title

\label{Anexos} % For referencing this appendix elsewhere, use \ref{AppendixA}

\section{Resultados del experimento preliminar con el constructivo por adyacentes}
\label{anex:tabla_adys}

En la tabla \ref{table:ax-pre-ady} se muestran los resultados tras analizar las instancias seleccionadas para la experimentación preliminar mediante el constructivo por adyacentes, separada por el uso del algoritmo \gls{GRASP} y la posterior adición de la mejora, donde:
\begin{itemize}
	\item $f$: Indica el valor de la función objetivo calculado.
	\item c: Indica la cardinalidad del clique encontrado.
	\item t: Indica el tiempo total empleado en encontrar la solución en segundos.
	\item desv (\%): Indica la variación respecto al valor de la función objetivo mediante el algoritmo MS-VNS.
	\item \# Mejor: Indica si la solución obtenida mediante el algoritmo es mejor o no que la del trabajo anterior.
\end{itemize}

\begin{scriptsize}
	\pgfplotstabletypeset[
	multicolumn names,
	empty header,
	begin table=\begin{longtable},
		every first row/.append style={before row={
				\caption{Resultados de los experimentos preliminares con constructivo adyacentes.}
				\label{table:ax-pre-ady}\\\toprule
				\multicolumn{1}{c|}{\textbf{Instancias}} &
				\multicolumn{5}{c}{\textbf{GRASP}} &
				\multicolumn{5}{|c}{\textbf{GRASP y búsqueda local}} \\
				\multicolumn{1}{c|}{} & \multicolumn{1}{c}{\textbf{$f$}} & \textbf{c} &\textbf{t (sec)} & \textbf{desv(\%)} &\textbf{\#Mejor} & \multicolumn{1}{|c}{\textbf{$f$}} & \textbf{c} & \textbf{t (sec)} & \textbf{desv(\%)} & \textbf{\#Mejor} \\ \toprule
				\endfirsthead
				%
				\multicolumn{11}{c}
				{{\bfseries Tabla \thetable\ Continuación de la página anterior}} \\
				\toprule 
				\multicolumn{1}{c|}{\textbf{Instancias}} &
				\multicolumn{5}{c}{\textbf{GRASP}} &
				\multicolumn{5}{|c}{\textbf{GRASP y búsqueda local}} \\
				\multicolumn{1}{c|}{} & \multicolumn{1}{c}{\textbf{$f$}} & \textbf{c} &\textbf{t (sec)} & \textbf{desv(\%)} &\textbf{\#Mejor} & \multicolumn{1}{|c}{\textbf{$f$}} & \textbf{c} & \textbf{t (sec)} & \textbf{desv(\%)} & \textbf{\#Mejor} \\ \toprule
				\endhead
				\midrule \multicolumn{11}{r}{{Continúa en la siguiente página}} \\ \bottomrule
				\endfoot
				\midrule
				\multicolumn{11}{r}{{Concluido}} \\ \bottomrule
				\endlastfoot
		}},
		end table=\end{longtable},
	col sep=semicolon,
	string type,
	display columns/0/.style={postproc cell content/.append style={@cell content={\textbf{##1}}}},
	display columns/1/.style={column type={S}},
	display columns/2/.style={column type={c}},
	display columns/3/.style={column type={S}},
	display columns/4/.style={column type={S}},
	display columns/5/.style={column type={c}},
	display columns/6/.style={column type={S}},
	display columns/7/.style={column type={S}},
	display columns/8/.style={column type={S}},
	display columns/9/.style={column type={S}},
	display columns/10/.style={column type={c}},
	]{Results/pre-ady-fin.csv}
\end{scriptsize}

\section{Resultados del experimento preliminar con el constructivo por ratio}
\label{anex:tabla_ratio} % For referencing this appendix elsewhere, use \ref{AppendixA}

En la tabla \ref{table:ax-pre-ratio} se muestran los resultados tras analizar las instancias seleccionadas para la experimentación preliminar mediante el constructivo de ratio, separadas por el uso del algoritmo \gls{GRASP} y su posterior mejora, donde:
\begin{itemize}
	\item $f$: Indica el valor de la función objetivo calculado.
	\item c: Indica la cardinalidad del clique encontrado.
	\item t: Indica el tiempo total empleado en encontrar la solución en segundos.
	\item desv (\%): Indica la variación respecto al valor de la función objetivo mediante el algoritmo MS-VNS.
	\item \# Mejor: Indica si la solución obtenida mediante el algoritmo es mejor o no que la del trabajo anterior.
\end{itemize}

\begin{scriptsize}
	\pgfplotstabletypeset[
	multicolumn names,
	empty header,
	begin table=\begin{longtable},
	every first row/.append style={before row={
			\caption{Resultados de los experimentos preliminares con constructivo ratio.}
			\label{table:ax-pre-ratio}\\\toprule
			\multicolumn{1}{c|}{\textbf{Instancias}} &
			\multicolumn{5}{c}{\textbf{GRASP}} &
			\multicolumn{5}{|c}{\textbf{GRASP y búsqueda local}} \\
			\multicolumn{1}{c|}{} & \multicolumn{1}{c}{\textbf{$f$}} & \textbf{c} &\textbf{t (sec)} & \textbf{desv(\%)} &\textbf{\#Mejor} & \multicolumn{1}{|c}{\textbf{$f$}} & \textbf{c} & \textbf{t (sec)} & \textbf{desv(\%)} & \textbf{\#Mejor} \\ \toprule
			\endfirsthead
			%
			\multicolumn{11}{c}
			{{\bfseries Tabla \thetable\ Continuación de la página anterior}} \\
			\toprule 
			\multicolumn{1}{c|}{\textbf{Instancias}} &
			\multicolumn{5}{c}{\textbf{GRASP}} &
			\multicolumn{5}{|c}{\textbf{GRASP y búsqueda local}} \\
			\multicolumn{1}{c|}{} & \multicolumn{1}{c}{\textbf{$f$}} & \textbf{c} &\textbf{t (sec)} & \textbf{desv(\%)} &\textbf{\#Mejor} & \multicolumn{1}{|c}{\textbf{$f$}} & \textbf{c} & \textbf{t (sec)} & \textbf{desv(\%)} & \textbf{\#Mejor} \\ \toprule
			\endhead
			\midrule \multicolumn{11}{r}{{Continúa en la siguiente página}} \\ \bottomrule
			\endfoot
			\midrule
			\multicolumn{11}{r}{{Concluido}} \\ \bottomrule
			\endlastfoot
	}},
	end table=\end{longtable},
	col sep=semicolon,
	string type,
	display columns/0/.style={postproc cell content/.append style={@cell content={\textbf{##1}}}},
	display columns/1/.style={column type={S}},
	display columns/2/.style={column type={c}},
	display columns/3/.style={column type={S}},
	display columns/4/.style={column type={S}},
	display columns/5/.style={column type={c}},
	display columns/6/.style={column type={S}},
	display columns/7/.style={column type={S}},
	display columns/8/.style={column type={S}},
	display columns/9/.style={column type={S}},
	display columns/10/.style={column type={c}},
	]{Results/pre-ratio-fin.csv}
\end{scriptsize}

\section{Resultados del experimento final}
\label{anex:tabla_final} % For referencing this appendix elsewhere, use \ref{AppendixA}

En la tabla \ref{table:ax-exp-final} se muestran los resultados tras analizar todas las instancias durante la experimentación final, donde:
\begin{itemize}
	\item $f$: Indica el valor de la función objetivo calculado.
	\item c: Indica la cardinalidad del clique encontrado.
	\item t: Indica el tiempo total empleado en encontrar la solución en segundos.
	\item desv(\#): Indica la variación respecto al valor de la función objetivo en la que se ha basado este trabajo.
	\item \#Mejor: Indica si la solución obtenida mediante el algoritmo es mejor o no que la del trabajo anterior.
\end{itemize}

\begin{footnotesize}
	\pgfplotstabletypeset[
	empty header,
	begin table=\begin{longtable},
		every first row/.append style={before row={%
				\caption{Resultados del experimento final.}
				\label{table:ax-exp-final}\\\toprule
				\textbf{Instancias} &\textbf{$f$} &\textbf{c} &\textbf{t (sec)} &\textbf{desv(\%)} &\textbf{\#Mejor} \\ \toprule    
				\endfirsthead
				\multicolumn{6}{c}%
				{{\bfseries Tabla \thetable\ Continuación de la página anterior}} \\
				\toprule 
				\textbf{Instancias} &\textbf{$f$} &\textbf{c} &\textbf{t (sec)} &\textbf{desv(\%)} &\textbf{\#Mejor} \\ \toprule    
				\endhead
				\midrule \multicolumn{6}{r}{{Continúa en la siguiente página}} \\ \bottomrule
				\endfoot
				\midrule
				\multicolumn{6}{r}{{Concluido}} \\ \bottomrule
				\endlastfoot
		}},
		end table=\end{longtable},
	col sep=semicolon,
	string type,
	display columns/0/.style={postproc cell content/.append style={@cell content={\textbf{##1}}}},
	display columns/1/.style={column type={S}},
	display columns/3/.style={column type={S}},
	display columns/4/.style={column type={S}},
	]{Results/tabla-final.csv}
\end{footnotesize}

%-------------------------------------------------------------------------------
%\include{Appendices/AppendixB}
%\include{Appendices/AppendixC}

%-------------------------------------------------------------------------------
%	THESIS CONTENT - CONCEPTS INDEX
%-------------------------------------------------------------------------------

\printindex

%-------------------------------------------------------------------------------
%	BIBLIOGRAPHY
%-------------------------------------------------------------------------------

\printbibliography[heading=bibintoc]

%-------------------------------------------------------------------------------

\end{document}
