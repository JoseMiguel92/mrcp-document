% Chapter 4: Design and Implementation

\chapter{Descripción informática} % Main chapter title

\label{Chapter4} % Reference

%-------------------------------------------------------------------------------

En este capítulo se describe el desarrollo completo del proyecto, desde el diseño  hasta la implementación de este, así como la metodología utilizada para la correcta evolución del mismo.

\section{Diseño}
aaa

\section{Implementación}
bbbb

\section{Metadología empleada}
Para el desarrollo de este proyecto se ha elegido por una metodología ágil ya que como afirma Joaquín Alviz (2016) \cite{Reference1} "para poder contar con versiones del producto funcional que se puedan mostrar al cliente conforme se va mejorando y completando, es necesario contar con un enfoque ágil, flexible y con una retroalimentación constante", por esto para la elaboración de este trabajo final de grado se ha optado por seguir una metodología ágil de tipo iterativa e incremental, lo que permite evolucionar el proyecto progresivamente e ir adaptando los requisitos del cliente, en este caso los tutores del proyecto, en el menor tiempo posible, mejorando así la calidad del producto final con el menor esfuerzo.

Estas iteraciones e incrementos de funcionalidad se han realizado durante todo el desarrollo del proyecto, mediante reuniones, con un lapso de aproximadamente 3 a 4 semanas entre ellas, corrigiendo errores de la iteración anterior si los hubiera y aumentando la funcionalidad del producto a entregar tanto en funcionalidad como en calidad, de esta manera se consigue una evolución progresiva y segura sobre el producto y los requerimientos que el proyecto exige.

Para mantener el control y administrar lo correspondiente sobre las tareas a realizar, las que se han realizado y las realizadas del proyecto, se ha usado la utilidad Trello\footnote{https://trello.com/es}, la cuál mediante tarjetas sobre el tablero del proyecto permite conocer las tareas del proyecto, así como conocer su estado, y añadir nuevas si así fuera necesario.

En cuanto al mantenimiento de versiones del proyecto se ha usado el sistema de control de versiones Git\footnote{https://git-scm.com/} a través del portal de alojamiento de repositorios GitHub\footnote{https://github.com/}, para interactuar entre el repositorio local y el repositorio remoto se ha optado por hacer uso tanto de la terminal mediante los comandos del propio sistema de control de versiones Git como del cliente para tal propósito GitKraken\footnote{https://www.gitkraken.com/}, el cuál permite mediante su sencilla e intuitiva interfaz mantener un control exhaustivo sobre las ramas y versionado de las distintas piezas de código del proyecto en el que se trabaja, así como revisar posibles conflictos que se produzcan.
%-------------------------------------------------------------------------------

