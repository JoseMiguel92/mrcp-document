% Chapter 5: Results

\chapter{Resultados} % Main chapter title

\label{Chapter5} % Reference

%-------------------------------------------------------------------------------

En este capítulo se exponen los diferentes recursos, tanto hardware como software, para la realización de este trabajo fin de grado y cómo, a partir de estos, se han obtenido los resultados para su posterior análisis.

\section{Recursos utilizados}
A continuación, se detallará la máquina y software empleados para el desarrollo del código, así como las instancias utilizadas para comprobar la calidad del algoritmo.

\subsection{Descripción de la máquina utilizada}
\label{sec:maquina}
Para la realización de las diversas pruebas y procesado de las instancias de este problema se ha utilizado una máquina con las siguientes características:

\begin{itemize}

\item \textbf{Procesador:} Intel(R) Core(TM) i5-5257U CPU 2.70 GHz
\item \textbf{Memoria RAM:} 8 GB 1867 Mhz DDR3
\end{itemize}

El desarrollo del código se ha realizado mediante el lenguaje de programación Python\footnote{https://www.python.org/}, en su versión 3.7.4, a través del entorno de desarrollo integrado o \gls{IDE} (Integrated Development Environment) PyCharm\footnote{https://www.jetbrains.com/es-es/pycharm/} de JetBrains en su versión 2019.3.1.

\subsection{Instancias utilizadas}
\label{sec:Instancias-utilizadas}
Las instancias con las que se ha contado para comprobar la eficiencia del algoritmo desarrollado han sido proporcionadas por los estudios previos en los que se basa y compara este trabajo final de grado. Estas hacen referencia a conjuntos de grafos, tanto generados de manera aleatoria como obtenidos de diferentes fuentes de datos, como precios del mercado de valores o turbinas de viento.

A continuación, se detallan los diferentes conjuntos en los que se dividen las instancias:

\begin{itemize}
	
	\item \textbf{Conjuntos de tipo A y B:} Estos conjuntos son instancias de grafos generadas mediante una distribución de probabilidad uniforme, variando su número de nodos entre 100 y 500, la densidad de estos grafos oscila entre el 45,78 \% y el 53,64 \%.
	\item  \textbf{Conjunto de tipo C:} Los datos pertenecientes a las instancias de este tipo hacen referencia a datos de los precios del mercado de valores.
	\item  \textbf{Conjunto de tipo D:} Las instancias de este conjunto son datos para la construcción de turbinas de viento, donde cada nodo representa una localización de estas turbinas y sus pesos son la media de la velocidad del viento y el coste de construcción de una turbina en ese punto.
	\item  \textbf{Conjunto de tipo E:} Estas instancias están extraídas del segundo y décimo DIMACS Implementation Challenge\footnote{http://dimacs.rutgers.edu/programs/challenge/}, donde cada nodo tiene un peso $\mathrm{p_{i} = 1}$ y un peso $\mathrm{q_{i} = 2}$. Adicionalmente se ha añadido un nodo más a cada instancia de este conjunto, el cual está conectado al resto de nodos de esta, con un peso $\mathrm{p_{i} = 1}$ y un peso $\mathrm{q_{i} = 1}$, donde i es el número del nodo dentro de esa instancia.
	\item  \textbf{Conjunto de tipo F:} Estas instancias son las mismas que en el conjunto E pero en este caso los pesos de cada nodo son $\mathrm{p_{i} = i}$ y $\mathrm{q_{i} = |V| - i + 1}$, donde i es el número del nodo dentro de esa instancia.
	
	
\end{itemize}

%-------------------------------------------------------------------------------

\section{Análisis de los resultados}
En esta sección se muestran los resultados obtenidos tras el procesamiento de las instancias descritas en la sección \ref{sec:Instancias-utilizadas} y empleando la máquina descrita en la sección \ref{sec:maquina}. Estos resultados se han obtenido en dos fases, una primera fase preliminar, con un conjunto reducido de las instancias y una segunda fase, con todas las instancias del problema para su posterior comparación con los resultados de estudios previos.

\subsection{Experimentos preliminares}
En esta fase ha sido seleccionado un subconjunto de 22 instancias del problema, con el fin de estudiar la calidad del algoritmo y comprobar los resultados obtenidos mediante distintos valores para $\alpha$ en varias iteraciones, estas se han elegido de todos los conjuntos para ampliar el rango de las pruebas. El número de nodos en estas instancias varia desde los 30 hasta los 1000 nodos, con densidades del 1 al 50 por ciento aproximadamente.

A esta selección de instancias se les ha aplicado los dos constructivos desarrollados, descritos en la sección \ref{sec:implementacion}, y a su vez se han aplicado los valores de $\alpha$ $\epsilon$ \{0.25, 0.5, 0.75\} y adicionalmente un valor de $\alpha$ aleatorio.
Este experimento se ha realizado procesando cada instancia 100 veces, que, unido a lo anteriormente descrito, hace un total de 800 iteraciones por fichero.

En las tablas \ref{tab:pre-ady} y \ref{tab:pre-ratio} se muestran los datos resumidos que se han recopilado durante todo el procesado preliminar. Estos han sido comparados con el análisis previo mediante el algoritmo MS-VNS, donde:

\begin{itemize}
	\item $f$: Indica el valor medio de la función objetivo.
	\item t(sec): Indica el tiempo medio empleado en el procesado de las instancias.
	\item Desv.(\%): Indica el porcentaje de desviación de la función objetivo respecto al análisis previo.
	\item \# Mejor: Indica el número de veces que ese constructivo ha sido mejor que el análisis previo.
\end{itemize}

Por un lado, la tabla \ref{tab:pre-ady} muestra los datos para el constructivo por adyacentes, se percibe que el valor medio de $f$ se acerca mucho al que se obtuvo en el análisis previo del problema, esto indica que el algoritmo cumple cierta calidad en la obtención de una solución factible.

\begin{table}[H]
	\centering
	\begin{tabular}{lllll}
		\hline
		& \textbf{$f$} & \textbf{t (sec)} & \textbf{Desv.(\%)} & \textbf{\# Mejor} \\ \hline
		\textbf{MS-VNS}     & 8789,20    & 31,42            & 0,21             & 21                \\
		\textbf{GRASP}      & 8074,21    & 4,31             & 48,16            & 1                 \\
		\textbf{GRASP + BL} & 8202,39    & 5,42             & 45,19            & 1                 \\ \hline
	\end{tabular}
	\caption{Tabla resumen constructivo de adyacentes.}
	\label{tab:pre-ady}
\end{table}

Por otro lado, en la tabla \ref{tab:pre-ratio}, se exponen los resultados para el constructivo por ratio, que en este caso consigue alcanzar mejor promedio para $f$. Cabe añadir que incluir la mejora aumenta en ambos casos el tiempo de cómputo. Sin embargo, también aumenta significativamente el valor de la función objetivo, por lo que se añadirá a la experimentación final.\\

\begin{table}[H]
	\centering
	\begin{tabular}{lllll}
		\hline
		& \textbf{f} & \textbf{t (sec)} & \textbf{Desv (\%)} & \textbf{\# Mejor} \\ \hline
		\textbf{MS-VNS}     & 8789,20    & 31,42            & 0,00             & 22                \\
		\textbf{GRASP}      & 8139,99    & 5,00             & 47,79            & 0                 \\
		\textbf{GRASP + BL} & 8202,43    & 5,89             & 45,35            & 0                 \\ \hline
	\end{tabular}
	\caption{Tabla resumen constructivo de ratio.}
	\label{tab:pre-ratio}
\end{table}

Aunque el constructivo mediante el ratio no alcance un valor que supere la función objetivo, se han obtenido de media mejores resultados en relación ratio-cardinalidad, con una desviación menor respecto al algoritmo MS-VNS. Por lo tanto, el constructivo basado en el ratio permite obtener valores para la función objetivo mayores y cliques con mayor número de nodos. A esto hay que añadir que el valor de $\alpha$ con el que se han obtenido mejores resultados ha sido 0.75, llegando a 16 soluciones mejores de las 22 instancias procesadas en este caso respecto del constructivo por adyacentes.

Todos los datos que se han recopilado durante estos experimentos preliminares se pueden encontrar en los anexos \ref{table:ax-pre-ady} y \ref{table:ax-pre-ratio}. En estas tablas se encuentran los mejores resultados tras todas las iteraciones de ambos constructivos y la posterior mejora del algoritmo con la búsqueda local.

\subsection{Experimento final}

En esta sección se han recopilado todos los datos obtenidos en las pruebas preliminares para el procesado final y se ha configurado de la siguiente manera:
\begin{itemize}
	\item Algoritmo \gls{GRASP}.
	\item Búsqueda local.
	\item Constructivo calculado por ratio.
	\item Valor de $\alpha = 0.75$. 
	\item 100 iteraciones por instancia.
\end{itemize}

Los datos recopilados durante el experimento final han sido comparados con los obtenidos previamente en el trabajo realizado por Dominik Goeke, Mahdi Moeini y David Poganiuch \cite{mrcp-GOEKE2017283}, los cuales se pueden observar en la tabla 2 del mismo. 

En la tabla \ref{tab:final-corregida} se muestran los resultados resumidos tras el procesado de todas las instancias, se puede apreciar que el valor promedio de $f$ es muy cercano al valor promedio que se obtuvo en el análisis de referencia.

\begin{table}[H]
	\centering
	\begin{tabular}{lllll}
		\hline
		\textbf{}           & \textbf{f} & \textbf{t(sec)} & \textbf{Desv.(\%)} & \textbf{\# Mejor} \\ \hline
		\textbf{MS-VNS}     & 1700,88    & 23,46           & 5,28               & 102               \\
		\textbf{GRASP + BL} & 1671,81    & 6,09            & 37,73              & 7\\ \hline             
	\end{tabular}
	\caption{Tabla comparativa corregida de la experimentación final.}
	\label{tab:final-corregida}
\end{table}

Es importante indicar que en esta tabla no se han añadido los resultados que en el estudio previo de Dominik Goeke, Mahdi Moeini y David Poganiuch no incluyeron o no procesaron. Si se añaden estos valores, se obtiene la tabla \ref{tab:final-completa}, donde se puede observar que el valor medio de $f$ es mayor que en el estudio previo. También cabe añadir que este algoritmo ha obtenido los resultados en un tiempo medio muy bajo respecto del estudio previo. Además, aplicar este algoritmo y posterior mejora a supuesto la menor desviación, respecto del algoritmo MS-VNS, de todas las experimentaciones.

\begin{table}[H]
	\centering
	\begin{tabular}{lllll}
		\hline
		\textbf{}           & \textbf{f} & \textbf{t(sec)} & \textbf{Desv.(\%)} & \textbf{\# Mejor} \\ \hline
		\textbf{MS-VNS}     & 1700,88    & 23,46           & 5,28               & 102               \\
		\textbf{GRASP + BL} & 2384,61    & 6,84            & 35,75              & 13 \\ \hline               
	\end{tabular}
	\caption{Tabla comparativa de la experimentación final.}
	\label{tab:final-completa}
\end{table}

En la tabla \ref{table:ax-exp-final} que se anexa se pueden encontrar los datos tras el procesado de todas las instancias durante el experimento final.
%-------------------------------------------------------------------------------