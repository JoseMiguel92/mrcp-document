% Chapter 3: Algorithm description

\chapter{Descripción algorítmica} % Main chapter title

\label{Chapter3}

%-------------------------------------------------------------------------------
En este capítulo se describe el algoritmo metaheurístico utilizado, exponiendo todas sus características para la obtención de una solución al problema.

\section{Heurística y Metaheurística}
Se define heurística, según el Diccionario de la Real Academia de la Lengua Española \cite{rae-heuristica}, como ``técnica de la indagación y del descubrimiento'' y ``en algunas ciencias, manera de buscar la solución de un problema mediante métodos no rigurosos, como por tanteo, reglas empíricas, etc.''.

Aplicándolo a temas científicos se define como el proceso de creación de medios, estrategias y principios para alcanzar un objetivo eficaz al problema dado \cite{conceptodef-heuristica}, y en relación a esto, el término, fue acuñado por George Polya en su libro ``How to Solve It'' \cite{gpolya-book-1}, más tarde traducido a ``Cómo plantear y resolver problemas'' \cite{gpolya-book-2}.

Añadiendo al término heurística el prefijo ``meta'' procedente del griego, que significa ``más allá'' o ``nivel superior'', se puede definir metaheurística como el conjunto de procedimientos heurísticos combinados para obtener una solución, aunque no exacta si eficiente, a un problema que no tiene un algoritmo heurístico específico o su aplicación es ineficiente \cite{wiki-metaheuristica}. Este término lo acuño Fred Glover en sus trabajos sobre búsqueda tabú en 1986 \cite{fred-glover}.

Existen un gran variedad de algoritmo metaheurísticos que se pueden clasificar de diferentes maneras según el enfoque, en la figura \ref{fig:clasif-metahs} se muestra una de ellas basada en el número de soluciones.\\
\begin{figure}[H]
	\centering
\diagram{Trayectoriales}{
	- \diagram{Búsqueda\\local}{
		- Con memoria: Búsqueda Tabú (TS)\\ 
		- \diagram{Estocástica}{
			- Búsqueda local guiada (GLS) \\ 
			- Métodos ruidosos (NM) \\ 
			- Recocido simulado (SA)\\
			- Métodos de aceptación de\\ umbral (TAM)\\
		}\\
		- \diagram{Por \\entornos}{
			- Búsqueda de entorno variable (VNS)\\
			- Optimización parcial bajo condiciones\\ especiales de intensificación(POPMUSIC)\\
		}\\
	}\\
	- \diagram{Búsqueda\\iterativa}{
		- Búsqueda local iterativa (ILS)\\
		- Búsqueda por entorno adaptativo borroso (FANS\\
	}\\
	- \diagram{Búsqueda\\multi-arranque}{
		- Concentración heurística (HC)\\
		- (AMS)\\
		- Métodos multi-arranque (MSM)\\
		- Procedimientos de búsqueda voráz,\\ aleatorizada y adaptativa (GRASP)\\
	}\\
}
\diagram{Poblacionales}{
	- \diagram{Combinación\\de soluciones}{
		- \diagram{Inspiración\\evolutiva}{
			- Algoritmos culturales (CA)\\
			- Algoritmos genéticos (GA)\\
			- Algoritmos meméticos (MA)\\
		}\\
		- \diagram{Sin inspiración}{
			- Re-encadenamiento de\\ caminos (PR)\\
			- Búsqueda dispersa (SS)\\
		}\\
	}\\
	- \diagram{Movimientos}{
		- Optimización por colonias de hormigas (ACO)\\
		- Equipos asíncronos (AT)\\
		- Algoritmos de estimación de distribución (EDA)\\
		- Inteligencia de enjambre (SI)\\
	}\\
}
	\caption{Clasificación de metaheurísticas}
	\label{fig:clasif-metahs}
\end{figure}

Dos conceptos a tener en cuenta en algoritmo metaheurístico son la intensificación o explotación en una región, de modo que una alta intensificación hará que el algoritmo realice una búsqueda más exhaustiva en esa región, y la diversificación o exploración de nuevas regiones del espacio de soluciones \cite{libro-metaheuristicas}.

\section{Metaheurística GRASP\index{GRASP}}
El acrónimo GRASP se forma a partir de su siglas en inglés Greedy Randomized Adaptative Search Procedure o que en castellano es Procedimiento de búsqueda voráz aleatorizado y adaptativo, el término fue introducido por primera vez por Feo y Resende en 1995 en su artículo con el mismo nombre \cite{grasp-feo-resende}.

Este algoritmo se basa en el multi-arranque, dónde cada arranque es una iteración de un procedimiento que está constituido por dos partes bien diferenciadas, la fase constructiva, en la que se obtiene una solución de buena calidad, y una fase de mejora, en la que partiendo de la solución obtenida en la fase anterior, se intenta mejorar localmente \cite{libro-metaheuristicas}. 
En \cite{grasp-flightrecoveryproblem} \cite{grasp-parallel} \cite{grasp-weapon} \cite{grasp-empaquetado} \cite{grasp-ruta} \cite{grasp-vertex} se pueden encontrar diversos documentos en los que se tratan problemas aplicando la metaheurística GRASP.

En el algoritmo \ref{alg:grasp} se muestra el pseudocódigo de la metaheurística GRASP que se ha empleado para el desarrollo y obtención de una solución para este problema, dicho algoritmo se puede explicar aplicándolo al problema tratado de la siguiente manera:\\
Partiendo de un grafo $G=(V, E)$ donde $V$ son los vértices o nodos del grafo, y $E$ las aristas que unen estos nodos. Se toma un vértice $v$ aleatorio de entre los vértices del grafo. $v$ se incluye en la solución $S$ ya que cumple con las restricciones del problema, descritas en la sección XXX, a partir de $v$ se construye la lista de candidatos $CL$, definida como todos los nodos adyacentes a $v$ que forman parte de la lista de nodos del grafo de partida. A partir de este momento se toma un elemento de la lista de candidatos y con este se obtiene, mediante una función voráz el nodo con valor máximo y mínimo de esta función, con estos valores y un valor dado $\alpha$, que oscilará entre $0$ y $1$ y marca la aleatoriedad de la función de manera que si $\alpha$ es igual a 1 la función será menos aleatoria y se obtendría solo el mejor candidato, mientras que si es 0 la función es más aleatoria y se obtendrán todos los candidatos posibles, teniendo esto en cuenta, se obtiene el valor de $\mu$, el cuál pondrá el límite para la lista de candidatos restringida $RCL$, obtenida mediante la lista de candidatos $CL$ ordenada de mayor valor a menor valor tomando los primeros valores hasta alcanzar el valor de $\mu$. Tomando esta lista, se elegirá aleatoriamente un nodo de la misma, $u$, el cuál se añadirá a la lista solución $S$ y eliminándolo de la lista de candidatos junto con los nodos de la lista de candidatos que no son adyacentes a ese nodo. Este procedimiento será repetido hasta que la lista de candidatos este vacía, obteniéndose en ese momento la lista $S$ que conformará una solución.

\begin{algorithm}
	\SetAlgoLined
	$ v \gets rnd( V ) $ \\[0.2cm]
	$ S \gets \{ v \} $ \\[0.2cm]
	$ CL \gets \{u \in V : (u, v) \in E\} $ \\[0.2cm]
	\While{$|CL| \not= 0$}{
		$ \mathrm{g_{min}} \gets $ $ \smash{\displaystyle\min_{c \in CL}} \hspace{0.1cm} g(c) $ \\[0.2cm]
		$ \mathrm{g_{max}} \gets $ $ \smash{\displaystyle\max_{c \in CL}} \hspace{0.1cm} g(c) $ \\[0.2cm]
		$ \mu \gets  \mathrm{g_{max}} - \alpha ( \mathrm{g_{max}} - \mathrm{g_{min}} ) $ \\[0.2cm]
		$ RCL \gets \{ c \in CL : g(c) \geq \mu \}  $ \\[0.2cm]
		$ u \gets rnd (RCL) $ \\[0.2cm]
		$ S \gets \cup \hspace{0.1cm} \{ u \}$ \\[0.2cm]
		$ CL \gets CL \textbackslash \{ u \} \textbackslash \{ w : (u, w) \notin E \}$  \\[0.2cm]
	}
	\Return S
	\caption{Pseudocódigo algoritmo GRASP.}
	\label{alg:grasp}
\end{algorithm}

\subsection{Fase constructiva}

En esta fase se ha usado el algoritmo \ref{alg:const-voraz} el cuál es común a ambos constructivos y sólo se diferencian en como obtiene cada uno su mejor nodo a escoger para incluir en su solución, los cuales se definen a continuación.\\
\begin{algorithm}
	\SetAlgoLined
	$ S \gets \emptyset $  \\[0.2cm]
	$ Adyacentes \gets  SeleccionAdyacentes(nodo) $  \\[0.2cm]
	\While{$|Adyacentes| \not= 0$}{
		$ candidato \gets buscarMejor(adyacente) $  \\[0.2cm]
		\If{formaClique(candidato)}{
			$ Adyacentes \gets Adyacentes \cap SeleccionAdyacentes(candidato) $  \\[0.2cm]
			$ S \gets S \cup \{candidato\} $  \\[0.2cm]
		}\Else{
			$ Adyacentes \gets Adyacentes \textbackslash \{candidato\} $
		}
	}
	\Return S
	\caption{Contructivo voráz.}
	\label{alg:const-voraz}
\end{algorithm}

En primer lugar el algoritmo \ref{alg:const-voraz-ratio} para el constructivo que obtiene una solución mediante un algoritmo voraz buscando el mayor ratio de cada nodo adyacente al de partida.\\
\begin{algorithm}
	$ ratio \gets -1 $ \\[0.2cm]
	$ nodoElegido \gets NULO $ \\[0.2cm]
	\For{nodo}{
		$ ratioNodo \gets calcularRatio(nodo) $ \\[0.2cm]
		\If{ratioNodo >\hspace{0.1cm}  ratio}{
			$ ratio \gets ratioNodo $ \\[0.2cm]
			$ nodoElegido \gets nodo $ \\[0.2cm]
		}
	}
	\Return nodoElegido
	\caption{Pseudocódigo método buscarMejor de tipo ratio.}
	\label{alg:const-voraz-ratio}
\end{algorithm}


Y en segundo lugar, el algoritmo \ref{alg:const-voraz-ady} para el constructivo que obtiene una solución mediante un algoritmo voraz buscando el mayor número de vecinos de cada nodo adyacente al de partida.\\
\begin{algorithm}
	$ vecinos \gets -1 $ \\[0.2cm]
	$ nodoElegido \gets NULO $ \\[0.2cm]
	\For{nodo}{
		$ vecinosNodo \gets SeleccionAdyacentes(nodo) $ \\[0.2cm]
		\If{$ |vecinosNodo| $ >\hspace{0.1cm}  vecinos}{
			$ vecinos \gets |vecinosNodo|  $ \\[0.2cm]
			$ nodoElegido \gets nodo $ \\[0.2cm]
		}
	}
	\Return nodoElegido
	\caption{Pseudocódigo método buscarMejor de tipo adyacentes.}
	\label{alg:const-voraz-ady}
\end{algorithm}

\subsection{Fase de búsqueda}
Definición de la fase de búsqueda.


%-------------------------------------------------------------------------------

