% Chapter 3: Algorithm description

\chapter{Descripción algorítmica} % Main chapter title

\label{Chapter3}

%-------------------------------------------------------------------------------
En este capítulo se describe el algoritmo utilizado para el desarrollo de este trabajo final de grado, exponiendo todas sus características.

\section{Metaheurística}
Que es la metaheuristica cuales hay y por que se ha usado este algoritmo.

\subsection{GRASP\index{GRASP}}
Definición de la metaheurística utilizada.

El pseudocódigo del algoritmo GRASP que se ha empleado para el desarrollo en este proyecto es el siguiente:

\begin{algorithm}
	\SetAlgoLined
	$ v \gets rnd( V ) $ \\[0.2cm]
	$ S \gets \{ v \} $ \\[0.2cm]
	$ CL \gets \{u \in V : (u, v) \in E\} $ \\[0.2cm]
	\While{$|CL| \not= 0$}{
		$ \mathrm{g_{min}} \gets $ $ \smash{\displaystyle\min_{c \in CL}} \hspace{0.1cm} g(c) $ \\[0.2cm]
		$ \mathrm{g_{max}} \gets $ $ \smash{\displaystyle\max_{c \in CL}} \hspace{0.1cm} g(c) $ \\[0.2cm]
		$ \mu \gets  \mathrm{g_{max}} - \alpha ( \mathrm{g_{max}} - \mathrm{g_{min}} ) $ \\[0.2cm]
		$ RCL \gets \{ c \in CL : g(c) \geq \mu \}  $ \\[0.2cm]
		$ u \gets rnd (RCL) $ \\[0.2cm]
		$ S \gets \cup \hspace{0.1cm} \{ u \}$ \\[0.2cm]
		$ CL \gets CL \textbackslash \{ u \} \textbackslash \{ w : (u, w) \notin E \}$  \\[0.2cm]
	}
	\Return S
	\caption{Pseudocódigo algoritmo GRASP.}
\end{algorithm}

Para los métodos con los que es posible obtener las soluciones en la fase constructiva se tiene el siguiente pseudocódigo:

\begin{algorithm}
	\SetAlgoLined
	$ S \gets \emptyset $  \\[0.2cm]
	$ Adyacentes \gets  SeleccionAdyacentes(nodo) $  \\[0.2cm]
	\While{$|Adyacentes| \not= 0$}{
		$ candidato \gets buscarMejor(adyacente) $  \\[0.2cm]
		\If{formaClique(candidato)}{
			$ Adyacentes \gets Adyacentes \cap SeleccionAdyacentes(candidato) $  \\[0.2cm]
			$ S \gets S \cup \{candidato\} $  \\[0.2cm]
		}\Else{
			$ Adyacentes \gets Adyacentes \textbackslash \{candidato\} $
		}
	}
	\Return S
	\caption{Contructivo voráz}
\end{algorithm}

El cuál es común a ambos, solo se diferencian en como obtiene cada uno su mejor nodo a escoger para incluir en su solución, los cuales se pueden definir de la siguiente manera:


\begin{enumerate}
	\item
	El constructivo 1 el cual obtiene una solución mediante un algoritmo voraz buscando el mayor ratio de cada nodo adyacente al de partida es el siguiente:
	
	\begin{algorithm}
		$ ratio \gets -1 $ \\[0.2cm]
		$ nodoElegido \gets NULO $ \\[0.2cm]
		\For{nodo}{
			$ ratioNodo \gets calcularRatio(nodo) $ \\[0.2cm]
			\If{ratioNodo >\hspace{0.1cm}  ratio}{
				$ ratio \gets ratioNodo $ \\[0.2cm]
				$ nodoElegido \gets nodo $ \\[0.2cm]
			}
		}
		\Return nodoElegido
		\caption{Pseudocódigo método buscarMejor de tipo ratio.}
	\end{algorithm}	
	
	\item
	 El constructivo 2 el cual obtiene una solución mediante un algoritmo voraz buscando el mayor número de vecinos de cada nodo adyacente al de partida es el siguiente:
	
	\begin{algorithm}
		$ vecinos \gets -1 $ \\[0.2cm]
		$ nodoElegido \gets NULO $ \\[0.2cm]
		\For{nodo}{
			$ vecinosNodo \gets SeleccionAdyacentes(nodo) $ \\[0.2cm]
			\If{$ |vecinosNodo| $ >\hspace{0.1cm}  vecinos}{
				$ vecinos \gets |vecinosNodo|  $ \\[0.2cm]
				$ nodoElegido \gets nodo $ \\[0.2cm]
			}
		}
		\Return nodoElegido
		\caption{Pseudocódigo método buscarMejor de tipo adyacentes.}
	\end{algorithm}
	
\end{enumerate}

\subsubsection{Fase constructiva}
Definición de los constructivos.

\subsubsection{Fase de búsqueda}
Definición de la fase de búsqueda.


%-------------------------------------------------------------------------------

