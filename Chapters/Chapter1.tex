% Chapter 1: Introduction

\chapter{Introducción} % Main chapter title

\label{Chapter1}

%-------------------------------------------------------------------------------

% Define some commands to keep the formatting separated from the content
\newcommand{\keyword}[1]{\textbf{#1}}
\newcommand{\tabhead}[1]{\textbf{#1}}
\newcommand{\code}[1]{\texttt{#1}}
\newcommand{\file}[1]{\texttt{\bfseries#1}}
\newcommand{\option}[1]{\texttt{\itshape#1}}
\newcommand{\Mod}[1]{\ (\mathrm{mod}\ #1)}

%-------------------------------------------------------------------------------
En este capítulo se introduce el tema a tratar partiendo de conceptos previos que ayudaran al lector a entender mejor el desarrollo, siguiendo con la definición del problema y la motivación del mismo, y por último, se muestra una revisión del estado del arte relacionado con este problema.

\section{Conceptos previos}
aaaaa

\section{Definición y motivación del problema}
Este problema trata de encontrar una solución aproximada al MRCP por sus siglas en inglés Maximum Ratio Clique Problem. La solución se obtiene como un subgrafo completo o clique de ratio máximo de un grafo, para ello se realiza una aproximación lo más cercana posible a la solución óptima mediante un algoritmo GRASP por sus siglas en inglés Greedy Randomized Adaptive Search Procedure, es cercana ya que se este problema pertenece al conjunto de resolución en tiempo polinomial no determinista o NP-complejos (Nondeterministic Polynomial time).

\section{Estado del arte}
aaaaa

%-------------------------------------------------------------------------------

