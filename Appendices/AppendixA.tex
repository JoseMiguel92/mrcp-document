% Appendix A

\chapter{Anexos} % Main appendix title

\label{Anexos} % For referencing this appendix elsewhere, use \ref{AppendixA}

\section{Resultados del experimento preliminar con el constructivo por adyacentes}
\label{anex:tabla_adys}

En la tabla \ref{table:ax-pre-ady} se muestran los resultados tras analizar las instancias seleccionadas para la experimentación preliminar mediante el constructivo por adyacentes, separada por el uso del algoritmo \gls{GRASP} y la posterior adición de la mejora, donde:
\begin{itemize}
	\item $f$: Indica el valor de la función objetivo calculado.
	\item c: Indica la cardinalidad del clique encontrado.
	\item t: Indica el tiempo total empleado en encontrar la solución en segundos.
	\item desv (\%): Indica la variación respecto al valor de la función objetivo mediante el algoritmo MS-VNS.
	\item \# Mejor: Indica si la solución obtenida mediante el algoritmo es mejor o no que la del trabajo anterior.
\end{itemize}

\begin{scriptsize}
	\pgfplotstabletypeset[
	multicolumn names,
	empty header,
	begin table=\begin{longtable},
		every first row/.append style={before row={
				\caption{Resultados de los experimentos preliminares con constructivo adyacentes.}
				\label{table:ax-pre-ady}\\\toprule
				\multicolumn{1}{c|}{\textbf{Instancias}} &
				\multicolumn{5}{c}{\textbf{GRASP}} &
				\multicolumn{5}{|c}{\textbf{GRASP y búsqueda local}} \\
				\multicolumn{1}{c|}{} & \multicolumn{1}{c}{\textbf{$f$}} & \textbf{c} &\textbf{t (sec)} & \textbf{desv(\%)} &\textbf{\#Mejor} & \multicolumn{1}{|c}{\textbf{$f$}} & \textbf{c} & \textbf{t (sec)} & \textbf{desv(\%)} & \textbf{\#Mejor} \\ \toprule
				\endfirsthead
				%
				\multicolumn{11}{c}
				{{\bfseries Tabla \thetable\ Continuación de la página anterior}} \\
				\toprule 
				\multicolumn{1}{c|}{\textbf{Instancias}} &
				\multicolumn{5}{c}{\textbf{GRASP}} &
				\multicolumn{5}{|c}{\textbf{GRASP y búsqueda local}} \\
				\multicolumn{1}{c|}{} & \multicolumn{1}{c}{\textbf{$f$}} & \textbf{c} &\textbf{t (sec)} & \textbf{desv(\%)} &\textbf{\#Mejor} & \multicolumn{1}{|c}{\textbf{$f$}} & \textbf{c} & \textbf{t (sec)} & \textbf{desv(\%)} & \textbf{\#Mejor} \\ \toprule
				\endhead
				\midrule \multicolumn{11}{r}{{Continúa en la siguiente página}} \\ \bottomrule
				\endfoot
				\midrule
				\multicolumn{11}{r}{{Concluido}} \\ \bottomrule
				\endlastfoot
		}},
		end table=\end{longtable},
	col sep=semicolon,
	string type,
	display columns/0/.style={postproc cell content/.append style={@cell content={\textbf{##1}}}},
	display columns/1/.style={column type={S}},
	display columns/2/.style={column type={c}},
	display columns/3/.style={column type={S}},
	display columns/4/.style={column type={S}},
	display columns/5/.style={column type={c}},
	display columns/6/.style={column type={S}},
	display columns/7/.style={column type={S}},
	display columns/8/.style={column type={S}},
	display columns/9/.style={column type={S}},
	display columns/10/.style={column type={c}},
	]{Results/pre-ady-fin.csv}
\end{scriptsize}

\section{Resultados del experimento preliminar con el constructivo por ratio}
\label{anex:tabla_ratio} % For referencing this appendix elsewhere, use \ref{AppendixA}

En la tabla \ref{table:ax-pre-ratio} se muestran los resultados tras analizar las instancias seleccionadas para la experimentación preliminar mediante el constructivo de ratio, separadas por el uso del algoritmo \gls{GRASP} y su posterior mejora, donde:
\begin{itemize}
	\item $f$: Indica el valor de la función objetivo calculado.
	\item c: Indica la cardinalidad del clique encontrado.
	\item t: Indica el tiempo total empleado en encontrar la solución en segundos.
	\item desv (\%): Indica la variación respecto al valor de la función objetivo mediante el algoritmo MS-VNS.
	\item \# Mejor: Indica si la solución obtenida mediante el algoritmo es mejor o no que la del trabajo anterior.
\end{itemize}

\begin{scriptsize}
	\pgfplotstabletypeset[
	multicolumn names,
	empty header,
	begin table=\begin{longtable},
	every first row/.append style={before row={
			\caption{Resultados de los experimentos preliminares con constructivo ratio.}
			\label{table:ax-pre-ratio}\\\toprule
			\multicolumn{1}{c|}{\textbf{Instancias}} &
			\multicolumn{5}{c}{\textbf{GRASP}} &
			\multicolumn{5}{|c}{\textbf{GRASP y búsqueda local}} \\
			\multicolumn{1}{c|}{} & \multicolumn{1}{c}{\textbf{$f$}} & \textbf{c} &\textbf{t (sec)} & \textbf{desv(\%)} &\textbf{\#Mejor} & \multicolumn{1}{|c}{\textbf{$f$}} & \textbf{c} & \textbf{t (sec)} & \textbf{desv(\%)} & \textbf{\#Mejor} \\ \toprule
			\endfirsthead
			%
			\multicolumn{11}{c}
			{{\bfseries Tabla \thetable\ Continuación de la página anterior}} \\
			\toprule 
			\multicolumn{1}{c|}{\textbf{Instancias}} &
			\multicolumn{5}{c}{\textbf{GRASP}} &
			\multicolumn{5}{|c}{\textbf{GRASP y búsqueda local}} \\
			\multicolumn{1}{c|}{} & \multicolumn{1}{c}{\textbf{$f$}} & \textbf{c} &\textbf{t (sec)} & \textbf{desv(\%)} &\textbf{\#Mejor} & \multicolumn{1}{|c}{\textbf{$f$}} & \textbf{c} & \textbf{t (sec)} & \textbf{desv(\%)} & \textbf{\#Mejor} \\ \toprule
			\endhead
			\midrule \multicolumn{11}{r}{{Continúa en la siguiente página}} \\ \bottomrule
			\endfoot
			\midrule
			\multicolumn{11}{r}{{Concluido}} \\ \bottomrule
			\endlastfoot
	}},
	end table=\end{longtable},
	col sep=semicolon,
	string type,
	display columns/0/.style={postproc cell content/.append style={@cell content={\textbf{##1}}}},
	display columns/1/.style={column type={S}},
	display columns/2/.style={column type={c}},
	display columns/3/.style={column type={S}},
	display columns/4/.style={column type={S}},
	display columns/5/.style={column type={c}},
	display columns/6/.style={column type={S}},
	display columns/7/.style={column type={S}},
	display columns/8/.style={column type={S}},
	display columns/9/.style={column type={S}},
	display columns/10/.style={column type={c}},
	]{Results/pre-ratio-fin.csv}
\end{scriptsize}

\section{Resultados del experimento final}
\label{anex:tabla_final} % For referencing this appendix elsewhere, use \ref{AppendixA}

En la tabla \ref{table:ax-exp-final} se muestran los resultados tras analizar todas las instancias durante la experimentación final, donde:
\begin{itemize}
	\item $f$: Indica el valor de la función objetivo calculado.
	\item c: Indica la cardinalidad del clique encontrado.
	\item t: Indica el tiempo total empleado en encontrar la solución en segundos.
	\item desv(\#): Indica la variación respecto al valor de la función objetivo en la que se ha basado este trabajo.
	\item \#Mejor: Indica si la solución obtenida mediante el algoritmo es mejor o no que la del trabajo anterior.
\end{itemize}

\begin{footnotesize}
	\pgfplotstabletypeset[
	empty header,
	begin table=\begin{longtable},
		every first row/.append style={before row={%
				\caption{Resultados del experimento final.}
				\label{table:ax-exp-final}\\\toprule
				\textbf{Instancias} &\textbf{$f$} &\textbf{c} &\textbf{t (sec)} &\textbf{desv(\%)} &\textbf{\#Mejor} \\ \toprule    
				\endfirsthead
				\multicolumn{6}{c}%
				{{\bfseries Tabla \thetable\ Continuación de la página anterior}} \\
				\toprule 
				\textbf{Instancias} &\textbf{$f$} &\textbf{c} &\textbf{t (sec)} &\textbf{desv(\%)} &\textbf{\#Mejor} \\ \toprule    
				\endhead
				\midrule \multicolumn{6}{r}{{Continúa en la siguiente página}} \\ \bottomrule
				\endfoot
				\midrule
				\multicolumn{6}{r}{{Concluido}} \\ \bottomrule
				\endlastfoot
		}},
		end table=\end{longtable},
	col sep=semicolon,
	string type,
	display columns/0/.style={postproc cell content/.append style={@cell content={\textbf{##1}}}},
	display columns/1/.style={column type={S}},
	display columns/3/.style={column type={S}},
	display columns/4/.style={column type={S}},
	]{Results/tabla-final.csv}
\end{footnotesize}

%-------------------------------------------------------------------------------